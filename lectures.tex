% \documentclass[a4paper,10pt]{article}
% \documentclass[titleauthor]{mwart}
\documentclass[10pt,a4paper]{article}

\usepackage[intlimits]{mathtools}
\usepackage{amsfonts}
\usepackage{amssymb}
\usepackage{amsthm}
\usepackage{manfnt}

\usepackage[mathscr]{eucal}

\usepackage{graphicx}
\usepackage{geometry}

\usepackage{arabtex}

\usepackage[pdftex,bookmarks,hidelinks]{hyperref}

\usepackage[OT4]{fontenc}
\usepackage[utf8]{inputenc}
\usepackage{polski}
\usepackage{indentfirst}

\usepackage[stable]{footmisc}

% Numeracja wykładów (by Jarek)
% http://42.pl/tex/latex2e/spisy/spis.html
\makeatletter
 \newcounter{lecture}
 \newcommand*\l@lecture{\@dottedtocline{1}{0em}{1.5em}}
 \newcommand \lecture[1]
 {
  \refstepcounter{lecture}
  \addcontentsline{lol}{lecture}{Wykład \protect\numberline{\thelecture}{(#1)}}
  \section*{Wykład \thelecture\normalfont\hspace\fill #1\hrule}
 }
 \newcommand \listoflectures { \section*{Wykłady} \@starttoc{lol} }
\makeatother

\author{\Large{Na podstawie wykładu prof. J.Węglarza} \\ Skład: unvector}
\title{\textbf{Metody probabilistyczne}}

\newtheorem{definition}{Definicja}[section]
\newtheorem{theorem}{Twierdzenie}[section]

\renewcommand{\qedsymbol}{$\blacksquare$}

\newcommand{\Warning}{\marginpar{\dbend}\textbf{\textit{Uwaga: }}}
\newcommand{\dd}{\; \mathrm{d}}

\numberwithin{equation}{subsection}

\transtrue

\begin{document}

\maketitle


\lecture{30.09.2008}

\tableofcontents

\section{Zdarzenia i prawdopodobieństwo}
\subsection{Uwagi wstępne}
Obserwując rzeczywistość spotykamy się często ze \textbf{zjawiskami losowymi
(przypadkowymi)}. Cechą charakterystyczną takiego zjawiska jest fakt,
że~określony wynik (ilościowy lub jakościowy) doświadczenia (obserwacji)
związanego z tym zjawiskiem \textbf{może wystąpić lub nie}, po spełnieniu
danego zbioru warunków $W$.

\lecture{06.10.2008}
Przeciwieństwem zjawisk losowych są \textbf{zjawiska deterministyczne},
czyli takie, dla których wyżej wymieniony wynik \textbf{na pewno zajdzie}
lub \textbf{na pewno nie zajdzie} po spełnieniu zbioru warunków $W$ (zbiór $W$
jest tak wybrany, abyśmy mogli zapewnić powtarzalność wszystkich warunków).
Przykładem takiego zjawiska jest wrzenie wody, podgrzanej do $100\,^\circ\mathrm{C}$
pod ciśnieniem $760\:\mathrm{mm}\:\mathrm{Hg}$.
Przykładami zjawisk losowych są natomiast:
\begin{itemize}
  \item rzut monetą
  \item oddziaływanie cząstek materii
  \item szumy układów elektronicznych
  \item naprężenia konstrukcji
  \item ruch uliczny
  \item obsługa procesów w systemach komputerowych.
\end{itemize}

Warto ponadto podkreślić, że na poziomie kwantowym nie ma w ogóle zjawisk
deterministycznych.
W odniesieniu do zjawisk losowych pojawia się fundamentalne pytanie: jak często,
w długiej serii powtarzalnych doświadczeń (w sensie spełnienia zbioru warunków
$W$) pojawia się określony wynik?
Historycznie najstarszymi zjawiskami losowymi, w stosunku do których postawiono
takie pytanie, były gry hazardowe -- w szczególności rozpowszechniona
w~średniowieczu gra w kości\footnote{słowo \textit{hazard} pochodzi od \RL{al azar}
 - tłum.\ kostka}. Pierwsze (błędne) próby obliczenia wyników tej gry pochodzą
z X w.

Natomiast próby naukowego podejścia do odpowiedzi na to pytanie przypadają
na połowęr~VII~w.\ i~są związane z nazwiskami uczonych:
\begin{itemize}
  \item Blaise Pascal (1623-1662),
  \item Pierre de Fermat (1601-1665)\footnote{Wielkie twierdzenie Fermata:
      $\forall n\in\mathbb{N}:\,\neg\exists x,y,z\in\mathbb{C}_+:\,,x^n+y^n=z^n$.
      Dowód tego twierdzenia został podany dopiero w 1994 r.\ przez A.~J.~Wiles,
      jednak do tej pory nie istnieje dowód analityczny.},
  \item Jacob Bernoulli (1654-1705).
\end{itemize}


Próbowali oni w oparciu o aparat współczesnej matematyki podać
def.~prawdopodobieństwa i~wartości oczekiwanych, a przede wszystkim zauważyli,
że w wielokrotnie powtarzanych doświadczeniach związanych ze zjawiskami losowymi
pojawiają się wyraźne prawidłowości. Doprowadziło to do wyodrębnienia
\textbf{rachunku prawdopodobieństwa} jako \textit{działu matematyki zajmującego
  się badaniem prawidłowości w zakresie zjawisk losowych} (definicja). Zauważmy
od razu, że losowości (przypadkowości) nie należy utożsamiać z ogólnie rozumianą
niepewnością. Jest to tylko jedna, choć najbardziej rozpowszechniona,
z~możliwych dróg jej opisu i badania.

Dalsze pogłębianie aparatu analitycznego rachunku prawdopodobieństwa miało
miejsce w~XVII~w.\ i~na początku XIX w.\ i było związane z nazwiskami uczonych:
Abraham de~Moivre, Pierre Simon Laplace, Carl Friedrich Gauss (1777-1855),
Simeon Poisson.

Kolejny ważny etap rozwoju rachunku prawdopodobieństwa to koniec XIX~w.
\ i~początek XX w., a nazwiska jakie należy tu wymienić to:
Pafnutij Czebyszew, Aleksander Lapunow i Andriej Markow.

Jednak mimo wielu głębokich wyników rachunek prawdopodobieństwa nie stanowił
jeszcze wówczas ścisłej teorii matematycznej, a wiele pojęć, w~szczególności
zdarzenia i~prawdopodobieństwa, miało charakter intuicyjny. Przełom w~tym
względzie nastąpił w roku 1933, kiedy po wielu nieudanych próbach aksjomatyzacji
rachunku prawdopodobieństwa (m.in. Polak Hugon Steinhaus (1887-1972)\footnote{Hugon
Steinhaus był doktorantem wielkiego matematyka Dawida Hilberta (1862-1943),
który w~1900~r.\ sformułował Wielki Problem (23 problemy matematyczne). Steinhaus
w~1963~r.\ otrzymał doktora Honoris Causa Politechniki Poznańskiej.},
A. N. Kołmogorow (1903-1987) podał aksjomatykę opartą na teorii
miary.  Od tego momentu zdarzenie losowe i~prawdopodobieństwo stały się ściśle
sprecyzowanymi obiektami matematycznymi, a rachunek prawdopodobieństwa (teoria
prawdopodobieństwa) przyjął formę systemu dedukcyjnego. Wówczas również zostały
sworzone podstawy teorii procesów losowych, a gwałtowny rozwój prac
w~tym zakresie oraz ich zastosowań uczynił z rachunku prawdopodobieństwa jeden
z~głównych działów współczesnej matematyki.

\subsection{Zdarzenia i działania na nich}
Jak wspomnieliśmy, przedmiotem rachunku prawdopodobieństwa jest badanie
prawidłowości w~zakresie zjawisk losowych. W~tym celu niezbędne jest zbudowanie
modelu matematycznego tych zjawisk. Wymaga to z kolei sprecyzowania dwóch
podstawowych pojęć: zdarzenia losowego i~prawdopodobieństwa. Zacznijmy
od~pojęcia zdarzenia losowego, dalej w~skrócie zdarzenia.

Jak powiedzieliśmy, cechą charakterystyczną zjawisk losowych jest fakt,
że~związane z nimi wyniki doświadczeń (obserwacji) mogą wystąpić również
po spełnieniu określonego zbioru warunków. Właśnie taki \textit{hipotetyczny
 (ilościowy lub jakościowy) wynik doświadczenia przeprowadzonego w~określonych
warunkach nazywa~się w~rachunku prawdopodobieństwa} \textbf{zdarzeniem} (definicja).

Weźmy przykładowo rzut sześcienną kostką, na której ściankach znajdują~się
liczby od~1 do~6. Zdarzeniem jest tu oczywiście wystąpienie na górnej ściance
kostki jednej z tych liczb. Analizując ten przykład zauważmy dalej, że~te
zdarzenia~są najprostszymi (nierozkładalnymi na~prostsze) zdarzeniami, jakie
mogą wystąpić w wyniku rozpatrywanego doświadczenia. Te \textit{najprostsze,
nierozkładalne wyniki danego doświadczenia nazywamy} \textbf{zdarzeniami
elementarnymi} (definicja).

\textit{Zbiór wszystkich zdarzeń elementarnych dla danego doświadczenia} nazywać
będziemy \textbf{przestrzenią zdarzeń elementarnych} i oznaczać przez $\Omega$,
a jej elementy przez $\omega_1, \omega_2, \omega_3,\dotsc$ (z~różnymi indeksami).
Wracając do rozpatrywanego przykładu zauważmy, że:
\begin{itemize}
  \item $\Omega = \{1, 2, 3, 4, 5, 6\}$,
  \item zdarzenie polegające na wypadnięciu liczby parzystej ma postać
    $\{2,4,6\}$,
  \item zdarzenie polegające na wypadnięciu liczby $\geq5$ ma postać $\{5, 6\}$
  \item itd.
\end{itemize}

Zauważamy zatem, że zdarzenia są podzbiorami przestrzeni zdarzeń elementarnych
$\Omega$; będziemy je oznaczać dużymi literami $A, B, C,\dotsc$, a~zbiór
wszystkich zdarzeń oznaczymy przez $\mathscr{A}$ (ew. $\mathscr{F}$).

\Warning Zbiór $\mathscr{A}$ nie zawsze jest tożsamy ze~zbiorem wszystkich
podzbiorów zbioru $\Omega$.

W zbiorze $\mathscr{A}$ wyróżniamy dwa szczególne zdarzenia:
\begin{itemize}
  \item \textbf{zdarzenie pewne} - zdarzenie, które w~wyniku danego
    doświadczenia \textit{musi zajść}, czyli zdarzenie pokrywające~się
    ze~zbiorem $\Omega$.
  \item \textbf{zdarzenie niemożliwe} - zdarzenie, które w~wyniku danego
    doświadczenia \textit{nie~może zajść}, czyli zdarzenie pokrywające~się
    ze~zbiorem $\emptyset$.
\end{itemize}

Ponieważ, jak wyżej zaobserwowaliśmy, \textbf{istnieje ścisła analogia pomiędzy
darzeniami a~zbiorami}, więc podstawowe działania na zdarzeniach są~po prostu
działaniami na~zbiorach. Przykładowo mówimy, że~zdarzenie $A$
\textit{zawiera~się} w~zdarzeniu $B$ (czyli $A\subset B$), jeśli każde zdarzenie
elementarne ze~zbioru $A$ należy również do zbioru $B$; \textit{sumą} zdarzeń
$A$, $B$ (czyli $A\cup B$) nazywamy zdarzenie złożone ze zdarzeń elementarnych
należących do zdarzenia $A$ lub do zdarzenia $B$; \textit{iloczynem} zdarzeń
$A$, $B$ (czyli $A\cap B$) nazywamy zdarzenie złożone ze zdarzeń elementarnych
należących do~zdarzenia $A$ i~do~zdarzenia $B$, itd.
\begin{itemize}
  \item Mówimy, że zdarzenia $A$, $B$ są \textbf{rozłączne} (wyłączają się), jeśli
    \[
      A\cap B = \emptyset
    \]
  \item Mówimy, że zdarzenia $A$, $A'$ są \textbf{przeciwstawne}, jeśli
    \[
      (A\cup A' = \Omega)\land(A\cap A' = \emptyset)
    \]
  \item Mówimy, że zdarzenia $A_1, A_2, A_3,\dotsc$ tworzą \textbf{podział
      przestrzeni $\Omega$}, jeśli
    \[
      \left(\bigcup_{i}A_i = \Omega\right)\land\left(A_i\cap A_j = \emptyset, i\neq j\right)
    \]
\end{itemize}


\lecture{13.10.2008}
\subsection{Aksjomatyczna definicja prawdopodobieństwa (aksjomatyka rachunku
  prawdopodobieństwa)}
Z obserwacji zjawisk losowych wynika, że możliwość zajścia różnych zdarzeń jest
różna. Chcąc oceniać zdarzenia ze względu na tę możliwość, musimy zatem zdefiniować
pewną miarę tej możliwości, którą nazwiemy właśnie prawdopodobieństwem.

Historycznie istniały różne określenia prawdopodobieństwa, np. \textbf{określenie
  klasyczne} (Laplace'a), czy też \textbf{określenie geometryczne}. Określenia
te miały jednak podstawowe wady; w szczególnosci wymagały założenia, że~możliwość
(czyli prawdopodobieństwo) zajścia każdego zdarzenia elementarnego jest jednakowa,
a tym samym definiowały prawdopodobieństwo przez prawdopodobieństwo. Przede
wszystkim jednak nie rozwiązywały one podstawowego problemu, polegającego na tym,
że nie zawsze każdy podzbiór zbioru $\Omega$ można uznać za zdarzenie, jeśli
prawdopodobieństwo ma posiadać podstawową własność miary, a mianowicie własność
\textbf{przeliczalnej addytywności ($\sigma$-addytywności)}, stwierdzającą,
że~prawdopodobieństwo sumy \textbf{dowolnej} liczby parami rozłącznych zdarzeń
jest równe sumie prawdopodobieństw tych zdarzeń. Okazuje się, że gdybyśmy
w~przypadku nieprzeliczalnego zbioru $\Omega$ wzięli w charakterze zdarzeń
losowych zbiór wszystkich podzbiorów tego zbioru, to otrzymalibyśmy w~powyższym
sformułowaniu jedynie znak $\leq$ (od sumy tych zdarzeń).

Zauważmy, że problem powyższy ma charakter uniwersalny, niezależny od charakteru
elementów zbioru $\Omega$ i dotyczy jakiejkolwiek miary, określonej
na~podzbiorach tego zbioru. Jest to mianowicie problem tzw.~\textbf{teorii miary},
która wraz z~teorią całki (z którą jest ściśle związana) stanowi główny nurt
teorii funkcji rzeczywistych, będących z kolei podstawowym działem analizy
matematycznej, charakteryzującym się najsłabszymi założeniami odnośnie badanych
obiektów (tych funkcji). Żeby mówić o tym ściślej, musimy wprowadzić szereg
definicji.

\begin{definition}
  Klasę (rodzinę) $\mathscr{A}$ podzbiorów przestrzeni $\Omega$ nazywamy
  \textbf{zamkniętą ze~względu na~daną operację}, jeśli wynik tej operacji
  na~elementach klasy $\mathscr{A}$ również należy do~$\mathscr{A}$.
\end{definition}

\begin{definition}
  Klasę $\mathscr{A}$ podzbiorów przestrzeni $\Omega$ nazywamy \textbf{algebrą
    (ciałem) zbiorów}, jeśli:
  \begin{enumerate}
  \item[1$^\circ$] $(A, B \in\mathscr{A}) \Longrightarrow (A\cup B \in \mathscr{A})$
  \item[2$^\circ$] $(A\in\mathscr{A})\Longrightarrow (A'=\Omega \backslash A
    \in \mathscr{A})$
  \end{enumerate}
\end{definition}

Zauważmy, że:
\begin{itemize}
  \item $\Omega=A\cup A' \in \mathscr{A}$,
  \item $\emptyset=\Omega'\in\mathscr{A}$,
  \item $A\cap B\in\mathscr{A}$ (gdyż z prawa de Morgana $(A\cap B)'=A'\cup B'$),
  \item $A\backslash B=A\cap B' \in\mathscr{A}$.
\end{itemize}
Widzimy zatem, że~algebra zbiorów to~klasa zbiorów zamknięta ze~względu
na~wszystkie operacje skończone.

\begin{definition}
  Algebrę $\mathscr{A}$ podzbiorów przestrzeni $\Omega$ nazywamy
  \textbf{$\sigma$-algebrą (algebrą przeliczalnie addytywną)} zbiorów, jeśli:
  \begin{equation}
    \left(A_1, A_2,\dotsc \in \mathscr{A}\right) \Longrightarrow
    \left(\bigcup_{i=1}^{\infty}A_i\in\mathscr{A}\right)
  \end{equation}
\end{definition}
Z prawa de Morgana łatwo wykazać, że $(A_1, A_2,\dotsc\in \mathscr{A})
\Longrightarrow \left(\bigcap_{i=1}^{\infty}A_i\in\mathscr{A}\right)$. Zatem
$\sigma$-algebra zbiorów to~klasa zbiorów zamknięta ze~względu na~wszystkie
operacje przeliczalne.

Można wykazać następujące twierdzenie:
\begin{theorem}
  Dla dowolnej niepustej klasy $\mathscr{M}$ podzbiorów przestrzeni $\Omega$,
  istnieje najmniejsza $\sigma$-algebra podzbiorów tej przestrzeni,
  zawierająca klasę $\mathscr{M}$. Nazywa się ona \textbf{$\sigma$-algebrą
    generowaną przez klasę $\mathscr{M}$.}
\end{theorem}

Jak~wiadomo, iloczyn przeliczalnej liczby zbiorów otwartych nie~jest na~ogół
zbiorem otwartym, a suma przeliczalnej liczby zbiorów domkniętych nie~jest
na~ogół zbiorem domkniętym; natomiast podstawowym wymogiem użyteczności dalszych
rozważań jest~to, aby~rozpatrywana klasa podzbiorów przestrzeni $\Omega$ była
$\sigma$-algebrą, zawierającą wszystkie zbiory otwarte (a zatem i~domknięte).
Prowadzi to do kolejnej definicji.
\begin{definition}
  $\sigma$-algebrą podzbiorów przestrzeni $\Omega$ generowaną przez klasy
  wszystkich zbiorów otwartych (lub domkniętych) nazywamy \textbf{$\sigma$-algebrą
  zbiorów borelowskich}, a jej elementy \textbf{zbiorami borelowskimi}
  (Emil Borel, (1871-1956)).
\end{definition}

Możemy teraz zdefiniować pojęcie miary oraz pojęcie funkcji mierzalnej.
\begin{definition}
  \textbf{Miarą} nazywamy każdą nieujemną, przeliczalnie addytywną funkcję
  rzeczywistą, określoną na pewnej $\sigma$-algebrze $\mathscr{A}$ podzbiorów
  przestrzeni $\Omega$, tzn.~nieujemną funkcję rzeczywistą $\mu$, nierówną
  tożsamościowo nieskończoności taką, że:
  \begin{enumerate}
    \item $[(A_1,A_2,\dotsc\in\mathscr{A})\land(A_i\cap A_j=\emptyset,i\neq j)]
      \Longrightarrow \left[\mu(\bigcup_{i=1}^{\infty}A_i)=\sum_{i=1}^{\infty}\mu(A_i)\right]$
    \item $\mu(\emptyset)=0$
  \end{enumerate}
\end{definition}

\begin{definition}
  Uporządkowaną trójkę $(\Omega, \mathscr{A}, \mu)$ nazywać będziemy
  \textbf{przestrzenią z miarą} (\textbf{przestrzenią mierzalną}), a~elementy
  $\sigma$-algebry $\mathscr{A}$ \textbf{zbiorami mierzalnymi}.
\end{definition}

\begin{definition}
  Funkcję rzeczywistą $f$ nazywamy \textbf{mierzalną względem miary $\mu$}, jeśli:
  \begin{equation}
    \bigwedge_a\{\omega: f(\omega)<a\}\in \mathscr{A}
  \end{equation}
  Podobnie dla $>,\leq,\geq$. $\mathscr{A}$ jest $\sigma$-algebrą podzbiorów przestrzeni $\Omega$.
\end{definition}

Przykłady miar:
\begin{enumerate}
  \item dla podzbioru przestrzeni $R^k$, miarą jest $k$\dywiz wymiarowa objętość
    (dla przestrzeni liniowej będzie to długość, dla płaszczyzny - powierzchnia,
    dla przestrzeni trójwymiaowej objętość, itd.);
  \item w ośrodku naładowanym elektrycznie, miarą jest ładunek elektryczny
    podzbioru tego ośrodka;
  \item dla podzbiorów ośrodka materialnego, miarą jest masa podzbioru;
  \item jeśli $x_0\in\Omega$ i $\mathscr{A}$ jest $\sigma$-algebrą podzbiorów
    przestrzeni $\Omega$, to
    \[
      \mu: \mu(A) = \left\{ \begin{array}{ll}
      1 & \textrm{jeżeli $x_0\in A$}\\
      0 & \textrm{jeżeli $x_0\notin A$}
      \end{array} \right.
    \]
  \item miara Lebesgue'a, mająca fundamentalne znaczenie dla teorii całki.
\end{enumerate}

Podamy teraz kolejny przykład miary - prawdopodobieństwo (\textbf{aksjomatyka
  Kołmogorowa }(1933) - czyli aksjomatyczna definicja prawdopodobieństwa).

Niech $\Omega$ będzie dowolnym zbiorem. Elementy $\omega_1, \omega_2,
\omega_3dotsc\in\Omega$ nazywamy \textbf{zdarzeniami elementarnymi}. Niech
$\mathscr{A}$ będzie $\sigma$-algebrą podzbiorów przestrzeni $\Omega$. Elementy
$A_1, A_2, A_3\dotsc\in\mathscr{A}$ nazywamy \textbf{zdarzeniami losowymi}.
Dla~każdego zdarzenia $A\in\mathscr{A}$ określona jest liczba rzeczywista
$P(A)$, zwana prawdopodobieństwem zdarzenia A, spełniającą następujące aksjomaty:
\begin{enumerate}
  \item[  AI.] $0 \leq P(A) \leq 1$
  \item[ AII.] $P(\Omega) = 1$
  \item[AIII.] $[(A_1,A_2,\dotsc\in\mathscr{A})\land(A_i\cap A_j=\emptyset,i\neq j)]
    \Longrightarrow[P(\bigcup_{i=1}^{\infty}A_i)=\Sigma_{i=1}^{\infty}P(A_i)]$
\end{enumerate}


\lecture{20.10.2008}
Jeśli chodzi o $\sigma$-algebrę $\mathscr{A}$, to jeżeli przestrzeń $\Omega$
jest skończona lub przeliczalna, to jest nią ($\sigma$-algebrą) zbiór wszystkich
podzbiorów przestrzeni $\Omega$; jeżeli natomiast $\Omega$ jest nieprzeliczalna,
to w charakterze $\mathscr{A}$ weźmiemy $\sigma$-algebrę zbiorów borelowskich
(najmniejsza $\sigma$-algebra generowana przez zbiór otwarty (domknięty)).

\Warning W~świetle poprzednio podanych definicji, widzimy, że~aksjomatyka
Kołmogorowa definiuje miarę $P$ na elementach $\sigma$-algebry zdarzeń
$\mathscr{A}$. Jest to tzw. \textbf{miara unormowana} ($P(\Omega)=1,
P(\emptyset)=0$), która jest również nazywana \textbf{miarą probabilistyczną}.

Uporządkowaną trójkę $\left\lbrace\Omega,\mathscr{A},P\right\rbrace$ nazywać
będziemy \textbf{przestrzenią probabilistyczną}. Widzimy zatem, że~istnieje
ścisły związek pomiędzy rachunkiem (teorią) prawdopodobieństwa a~teorią miary.

\subsection{Prawdopodobieństwo warunkowe i całkowite. Twierdzenie Bayesa}

\begin{theorem}[O prawdopodobieństwie warunkowym]
  Jeśli $P(B)>0$, to prawdopodobieństwo warunkowe zdarzenia $A$ \textbf{pod
    warunkiem}, że zaszło zdarzenie $B$, wyraża się wzorem:

  \begin{equation}
    \label{eq:prob:dep}
    P(A|B) = \frac{P(A\cap B)}{P(B)}
  \end{equation}
\end{theorem}
\begin{proof}
  TODO
\end{proof}

Zauważmy, że $P(A\cap B)=P(B)\cdot P(A|B)$. Zatem jeśli $P(A_1\cap A_2)>0$, to:
\[
  P(A_3|A_1\cap A_2)=\frac{P(A_1\cap A_2 \cap A_3)}{P(A_1\cap A_2)}
\]
Zatem:
\[
  P(A_1\cap A_2 \cap A_3) = P(A_1\cap A_2)\cdot P(A_3|A_1\cap A_2) = P(A_1)
  \cdot P(A_2|A_1)\cdot P(A_3|A_1\cap A_2)
\]
i analogicznie dla dowolnego n.

\begin{theorem}[O prawdopodobieństwie całkowitym]
  Jeśli zdarzenia $A_1,A_2,\dotsc$ tworzą podział przestrzeni $\Omega$ oraz
  $P(A_i)>0$ dla $i=1,2,\dotsc$, to dla~dowolnego zdarzenia $B$ słuszny jest
  wzór:
  \begin{equation}
    \label{eq:prob:sum}
    P(B) = \sum_{i=1}^{\infty} P(A_i)\cdot P(B|A_i)
  \end{equation}
\end{theorem}
\begin{proof}
  Jeśli $A_1,A_2,\cdots$ stanowią podział przestrzeni $\Omega$ zachodzi:
  \[
    B = \bigcup_{i=1}^n B \cup A_i
  \]

  Z~czego wynika, że:

  \[
    P(B) = \sum_{i=1}^n P(A_i)P(B|A_i)
  \]

  Przekształcając wzór~\ref{eq:prob:dep} otrzymujemy $P(B \cap A_i) = P(A_i)
  P(B|A_i)$, co po podstawieniu do powyższego wzoru daje:

  \begin{equation}
    P(B) = \sum_{i=1}^{n} P(A_i)\cdot P(B|A_i)
  \end{equation}
\end{proof}

\begin{theorem}[Bayesa]
  Jeśli zdarzenia $A_1,A_2,\dotsc$ tworzą podział przestrzeni $\Omega$,
  $P(A_i)>0$ dla $i=1,2,\dotsc$ oraz $B$ jest zdarzeniem, dla którego $P(B)>0$,
  to słuszny jest następujący wzór:
  \begin{equation}
    P(A_i|B) = \frac{P(A_i)\cdot P(B|A_i)}{\sum_{j=1}^{\infty}P(A_j)\cdot
      P(B|A_j)},\:\textrm{dla}\:i=1,2,\dotsc
  \end{equation}
\end{theorem}
\begin{proof}
  TODO
\end{proof}

Powyższy wzór nazywa~się \textbf{wzorem Bayesa}, lub wzorem na~prawdopodobieństwo
\textbf{a~posteriori} (gdyż definiuje prawdopodobieństwa zdarzeń $A_i$
\textbf{po~zajściu} zdarzenia~$B$). Natomiast występujące w~tym wzorze
prawdopodobieństwa $P(A_i)$ nazywają się prawdopodobieństwami \textbf{a~priori}
lub prawdopodobieństwami \textbf{subiektywnymi}.

\Warning Wzór Bayesa ma~ważne zastosowanie w~teorii decyzji. Wówczas
$A_1,A_2,\dotsc$ oznaczają szeroko rozumiane stany natury, a~prawdopodobieństwa
$P(B|A_i)$~są wiarygodnością wyniku doświadczenia~$B$ dla~stanów $A_i$. Zauważmy,
że aby zastosować wzór Bayesa, musimy znać także prawdopodobieństwa a~priori
$P(A_i), i=1,2,\dotsc$. Tzw. \textbf{postulat Bayesa} zakłada właśnie
, że~prawdopodobieństwa te, tworzące tzw. \textbf{rozkład a~priori},
są~\textbf{zawsze} znane, niezależnie od tego, czy natura jest losowa,
czy~też~nie. Stanowią one mianowicie podsumowanie naszej wiedzy i/lub
intuicji o~badanej rzeczywistości. W tym kontekście prawdopodobieństwa
$P(A_i|B), i=1,2,\dotsc$, wyliczane ze wzoru Bayesa tworzą rozkład uściślający
tę~wiedzę i/lub intuicję.

\Warning Wzór Bayesa uogólnia się naturalnie
na~sytuację, gdy zamiast zdarzenia $B$ mamy zdarzenia $B_1,B_2,\dotsc$, tworzące
podział przestrzeni $\Omega$ i~takie, że $P(B_k)>0$ dla $k=1,2,\dotsc$; wówczas
wzór ten przyjmuje postać:
\begin{equation}
  P(A_i|B_k) = \frac{P(A_i)\cdot P(B_k|A_i)}{\sum_{j=1}^{\infty}P(A_j)\cdot
    P(B_k|A_j)},\:\textrm{dla}\:i=1,2,\dotsc,\,k=1,2,\dotsc
\end{equation}
\begin{proof}
  TODO
\end{proof}

\subsection{Niezależność zdarzeń}
Jeżeli prawdopodobieństwo zdarzenia~$A$ pod~warunkiem, że~zaszło zdarzenie~$B$
jest różne od~prawdopodobieństwa bezwarunkowego zdarzenia~$A$, to~znaczy,
że~zajście zdarzenia~$B$ dostarcza pewnej informacji o~zajściu zdarzenia $A$,
a~więc zdarzenia~te są~\textbf{zależne}.

Jeżeli natomiast prawdopodobieństwo warunkowe zdarzenia~$A$ pod warunkiem,
że~zaszło zdarzenie~$B$ jest równe prawdopodobieństwu bezwarunkowemu
zdarzenia~$A$, to powiemy, że zdarzenia $A$ i~$B$ są~\textbf{niezależne}.

Zauważmy, że zachodzi to, gdy:
\[
  P(A\cap B) = P(A)\cdot P(B),
\]
gdyż wówczas:
\[
  P(A|B)=\frac{P(A\cap B)}{P(B)}=\frac{P(A)\cdot P(B)}{P(B)} = P(A)
\]
Oczywiście, zachodzi również:
\[
  P(B|A)=\frac{P(A\cap B)}{P(A)}=\frac{P(A)\cdot P(B)}{P(A)} = P(B)
\]

Wynika stąd następująca definicja.

\begin{definition}
  Zdarzenia $A$ i $B$ nazywamy \textbf{niezależnymi}, jeśli
  \begin{equation}
    P(A\cap B)=P(A)\cdot P(B).
  \end{equation}
\end{definition}

Łatwo wykazać poniższe twierdzenie.
\begin{theorem}
  Następujące zdania są prawdziwe:
  \begin{enumerate}
    \item[1$^\circ$] dowolne zdarzenie~$A$ i~zdarzenie pewne ($\Omega$)
      są~niezależne
    \item[2$^\circ$] dowolne zdarzenie~$A$ i~zdarzenie niemożliwe ($\emptyset$)
      są~niezależne
    \item[3$^\circ$] zdarzenie pewne ($\Omega$) i~zdarzenie niemożliwe
      ($\emptyset$) są~niezależne
    \item[4$^\circ$] jeśli zdarzenia $A$ i~$B$ są niezależne, to~także zdarzenia
      $A$ i~$B'$ są~niezależne
  \end{enumerate}
\end{theorem}
\begin{proof}
  TODO
\end{proof}

Definicja niezależności zdarzeń uogólnia~się na~większą ich liczbę:
\begin{definition}
  Powiemy, że zdarzenia $A_1,A_2,...,A_n$ są~\textbf{niezależne}, jeśli
  prawdopodobieństwo iloczynu dowolnej liczby różnych zdarzeń spośród nich jest
  równe iloczynowi prawdopodobieństw tych zdarzeń.
\end{definition}

\Warning niektórzy nazywają w/w niezależność \textbf{niezależnością wzajemną}
(en~block) w~odróżnieniu od niezależności \textbf{parami}, \textbf{trójkami},
itd., gdy tylko pary, trójki, itd. są~niezależne. Oczywiście niezależność
wzajemna implikuje niezależność parami, trójkami, itd.,
\textbf{ale~nie~odwrotnie}.

\begin{definition}
  Zdarzenia $A_1,A_2,\dotsc$ są~niezależne, jeśli dla~dowolnego~$n$
  zdarzenia~$A_1,A_2,\dotsc,A_n$ są~niezależne.
\end{definition}

Łatwo wykazać kolejne twierdzenie:

\begin{theorem}
  Następujące zdania są prawdziwe:
  \begin{enumerate}
    \item jeśli zdarzenia $A,B,C$ są~niezależne, to~także zdarzenia $A\cup B$
      i~$C$ są niezależne
    \item jeśli zdarzenia $A,B,C$ są~niezależne oraz $P(C)>0$, to:
      \begin{equation}
        P(A\cap B|C)=P(A|C)\cdot P(B|C)
      \end{equation}
  \end{enumerate}
\end{theorem}
\begin{proof}
  TODO
\end{proof}


\section{Zmienne losowe}
\subsection{Pojęcie zmiennej losowej oraz jej rozkładu (prawdopodobieństwa)
  i~dystrybuanty}
Jak pamiętamy, natura zdarzeń elementarnych może być różnoraka,
np.~w~rozpatrywanym rzucie monetą $\Omega=\left\lbrace O,R\right\rbrace$,
w~rzucie kostką sześcienną $\Omega=\left\lbrace 1,2,3,4,5,6\right\rbrace$, itp.
Żeby~się od~tej różnorakości uwolnić, przypiszemy każdemu zdarzeniu
elementarnemu liczbę rzeczywistą, np.~w~pierwszym przypadku wyjściu orła 0,
a~wyjściu reszki 1, w~drugim przypadku tę liczbę, która się pojawiła. W~ten
sposób definiujemy na przestrzeni zdarzeń elementarnych $\Omega$ pewną funkcję
rzeczywistą. Chcemy jednak, aby dla tej funkcji było określone prawdopodobieństwo
przyjmowanych przez nią wartości z jak najobszerniejszej klasy podzbiorów prostych.
Wiemy, że taką klasę stanowi $\sigma$-algebra zbiorów borelowskich. Jeśli zatem
zażądamy, aby przeciwobrazy zbiorów borelowskich na prostej otrzymane za pomocą
definiowanej funkcji, były zdarzeniami (czyli elementami $\sigma$-algebry
$\mathscr{A}$), to prawdopodobieństwo takie będzie określone.

\lecture{27.10.2008}
Załóżmy, że dana jest przestrzeń probabilistyczna $\{\Omega,\mathscr{A},P\}$.

\begin{definition}
  \textbf{Zmienną losową} (\textbf{rzeczywistą}) nazywamy funkcję rzeczywistą~$X$
  określoną na przestrzeni zdarzeń elementarnych $\Omega$ i taką, że~przeciwobrazy
  (otrzymane za pomocą tej funkcji) zbiorów borelowskich na prostej (w~$\mathbb{R}^1$)
  są zdarzeniami losowymi.
\end{definition}

Wystarczy wymagać, by przeciwobrazy $(-\infty, x)$ były zdarzeniami losowymi,
gdyż każdy zbiór borelowski na prostej można otrzymać za pomocą przeliczalnej
liczby działań na~takich przedziałach.

Innymi słowy, zmienna losowa~$X$ to~funkcja $\Omega\rightarrow\mathbb{R}^1$
taka, że:
\begin{equation}
  \bigwedge_x X^{-1}[(-\infty, x)] = \{\omega: X(\omega) < x\}\in\mathscr{A}
\end{equation}
Zauważamy zatem, że zmienna losowa to funkcja odwzorowująca przestrzeń $\Omega$
w~prostą, mierzalna względem prawdopodobieństwa (czyli miary probabilistycznej~$P$).

W~skrócie można zatem powiedzieć, że zmienna losowa jest mierzalnym względem~$P$
odwzorowaniem $\Omega$ w prostej. Zmienne losowe będziemy oznaczać $X(\omega),
Y(\omega),\dotsc$ lub~w~skrócie $X, Y,\dotsc$, natomiast ich wartości (czyli
\textbf{realizacje}) $x, y,\dotsc$.

Z~powyższego wynika, że~zmienna losowa $X$ w~naturalny sposób generuje na~prostej
$\sigma$-algebrę zbiorów borelowskich, których przeciwobrazy są~zdarzeniami
losowymi, a~zatem $\sigma$-algebrę $\mathscr{A}_X$ zdefiniowaną wzorem:

\begin{equation}
  \mathscr{A}_X = \{S: X^{-1}(S)\in\mathscr{A}\}
\end{equation}

Generuje ona także prawdopodobieństwo~$P_X$ na~elementach $\sigma$-algebry
$\mathscr{A}_X$, zwane \textbf{prawdopodobieństwm indukowanym}, określone wzorem:

\begin{equation}
  P_X(S) = P[X^{-1}(S)], S\in\mathscr{A}_X
\end{equation}

Prawdopodobieństwo indukowane definiuje tzw.~\textbf{rozkład prawdopodobieństwa
  zmiennej losowej}, zdefiniowany zatem w~następujący sposób:

\begin{equation}
  P_X(S) = P[X^{-1}(S)] = P[\omega: X(\omega)\in S], S\in\mathscr{A}_X
\end{equation}

Oczywiście $P_X(\mathbb{R}^1) = P(\Omega) = 1$.

\Warning Często zmienna losowa odwzorowuje przestrzeń $\Omega$ w~pewien
podzbiór~$\mathbb{R}^1$. Wówczas można określić rozkład prawdopodobieństwa tylko na tym
podzbiorze; zwykle jednak dla~jednolitości dodaje~się do~tego podzbioru resztę
zbioru~$\mathbb{R}^1$, traktując~ją jako zdarzenie z prawdopodobieństwem równym $0$.

Jak widzimy, rozkład prawdopodobieństwa jest \textbf{funkcją zbioru}. W~klasycznej
analizie jesteśmy przyzwyczajeni do~\textbf{funkcji punktu}, chcąc zatem stosować
aparat tej analizy zadajmy pytanie o~istnienie funkcji punktu jednoznacznie
odpowiadającej rozkładowi prawdopodobieństwa. Na gruncie teorii miary jest~to
tzw.~\textbf{problem dystrybuanty miary}. Można bowiem wykazać, że~funkcja taka
zawsze istnieje i~nazywa się dystrybuantą.

Zdefiniujmy pewną funkcję rzeczywistą.

\begin{definition}
  $F(x)$ będzie funkcją rzeczywistą, zdefiniowaną wzorem:
  \begin{eqnarray}
    \label{eq:dist}
    F(x) & = & P_X[(-\infty, x)] = P\{X^{-1}[(-\infty, x)]\} =
    \nonumber \\
    & = &  P\{\omega: X(\omega)\in(-\infty,x)\} = P(X<x)
  \end{eqnarray}
\end{definition}

Łatwo zauważyć, że~funkcja $F(x)$ ma~następujące własności:

\begin{enumerate}
  \item \label{enum:dist:p1} jest nieujemna i~niemalejąca
  \item \label{enum:dist:p2} $F(-\infty) = 0$, $F(+\infty) = 1$
  \item \label{enum:dist:p3} jest przynajmniej lewostronnie ciągła,
    tzn.~dla~każdego ciągu $x_1<x_2<\dotsc<x$, zbieżnego do x zachodzi:
    \[
      \lim_{k\rightarrow\infty}F(x_k) = F(x)
    \]
\end{enumerate}

Zauważmy jeszcze, że:

\[
  \bigwedge_a P(X\geq a) = 1 - F(a)
\]
\[
  \bigwedge_{a,b: a<b} P(a\leq X<b) = F(b) - F(a)
\]

Można wykazać następujące twierdzenie:

\begin{theorem}
  Na~to, by~funkcja rzeczywista $F(x)$ była dystrybuantą zmiennej losowej~$X$
  potrzeba i~wystarcza, aby miała własności~\ref{enum:dist:p1},~\ref{enum:dist:p2}
  i~\ref{enum:dist:p3}.
\end{theorem}

Z~twierdzenia tego wynika, że~funkcja zdefiniowana wzorem~\ref{eq:dist} jest
dystrybuantą zmiennej losowej~$X$, a zatem jedno-jednoznacznie odpowiada
rozkładowi prawdopodobieństwa tej zmiennej.

Z~własności~\ref{enum:dist:p3} dystrybuanty wynika, że~może ona mieć punkty
nieciągłości. Można jednak wykazać, że~zbiór tych punktów jest co najwyżej
przeliczalny.

\subsection{Typy zmiennych losowych}
Zdefiniujemy obecnie dwa główne typy zmiennej losowej.

\begin{definition}
  Zmienną losową $X$ nazwiemy \textbf{typu skokowego} (lub~\textbf{dyskretną})
  jeśli może przyjmować wartości ze~zbioru co~najwyżej przeliczalnego.
  Wartości~te, które oznaczymy przez $x_i, i=1,2,\dotsc$, nazywają~się
  \textbf{punktami skokowymi}, a prawdopodobieństwa, z~którymi~są przyjmowane,
  i~które oznaczymy przez $p_i, i=1,2,\dotsc$, nazywają~się \textbf{skokami}
  (lub~\textbf{masami}).
\end{definition}

Oczywiście $p_i\geq0, i=1,2,\dotsc$, oraz $\sum_{i=1}^{\infty}p_i=1$. Natomiast
wzór $P(X=x_i)=p_i,i=1,2,\dotsc$ definiuje tzw. \textbf{funkcję
  prawdopodobieństwa} zmiennej losowej~$X$ typu skokowego. Oczywiście funkcja
ta~w~pełni charakteryzuje rozkład prawdopodobieństwa zmiennej losowej typu
skokowego. Jest również oczywiste, że dystrybuanta zmiennej losowej typu
skokowego przyjmuje postać:

\begin{equation}
  F(x) = P(X<x) = \sum_{x_i<x}p_i
\end{equation}

W domu: wykreśl dystrybuantę zmiennej losowej X, opisującej rzut sześcienną
kostką do gry (zaznacz punkty nieciągłości).

\begin{definition}
  Zmienną losową~$X$ nazwiemy \textbf{typu ciągłego}, jeśli istnieje funkcja
  rzeczywista $f(x)$ taka, że~dla~każdego $x$ zachodzi związek:
  \begin{equation}
    F(x)=\int_{-\infty}^{x}f(x)\dd{x}
  \end{equation}
  gdzie $F(x)$ jest dystrybuantą zmiennej losowej $X$, a~całkowanie odbywa~się
  w~sensie Lebesgue'a.
\end{definition}

Funkcja $f(x)$ zdefiniowana wyżej nazywa się \textbf{funkcją gęstości
prawdopodobieństwa} zmiennej losowej $X$ (ang. \textit{probability density
function} - PDF), \textbf{gęstością prawdopodobieństwa} lub krótką
\textbf{gęstością}.

Łatwo zauważyć, że funkcja $f(x)$ spełnia poniższe twierdzenie:

\begin{theorem}
  Funkcja gęstości prawdopodobieństwa zmiennej losowej $X$ ma następujące
  własności:
  \begin{itemize}
    \item $f(x) \geq 0$
    \item $\int_{-\infty}^\infty f(x)\dd{x} = F(+\infty) = 1$
  \end{itemize}
\end{theorem}

Można wykazać, że słuszne jest również twierdzenie odwrotne:

\begin{theorem}
  Każda funkcja rzeczywista $f(x)$ nieujemna, całkowalna na osi $X$ (w sensie
  Lebesgue'a) i taka, że $\int_{-\infty}^\infty f(x)\dd{x} = 1$, jest funkcją
  gęstości prawdopodobieństwa pewnej zmiennej losowej typu ciągłego $X$.
\end{theorem}

Widzimy zatem, że funkcja gęstości prawdopodobieństwa w pełni opisuje rozkład
prawdopodobieństwa zmiennej losowej typu ciągłego.

\begin{table}[ ht ]
  Podsumowanie:
  \vspace{0.2cm}

  \begin{minipage}[b]{0.45\linewidth}
    \begin{center}
      Zmienna typu skokowego
      \[
        F(x) = \sum_{x_i < x} p_i
      \]
      \[
        \bigwedge_i P(X = x_i) = p_i
      \]
      (funkcja prawdopodobieństwa)
    \end{center}
  \end{minipage}
  \hspace{0.5cm}
  \begin{minipage}[b]{0.45\linewidth}
    \begin{center}
      Zmienna typu ciągłego
      \[
        F(x) = \int_{-\infty}^{x} f(x) \dd{x}
      \]
      \[
        f(x)
      \]
      (funkcja gęstości prawdopodobieństwa)
    \end{center}
  \end{minipage}
\end{table}

\Warning Jeśli zmienna losowa nie ma punktów skokowych, to jej
dystrybuanta jest funkcją ciągłą. Nie oznacza to jednak, że odpowiada ona
zmiennej losowej typu ciągłego. Żeby tak było, dystrybuanta musi być funkcją
\textbf{absolutnie} (\textbf{bezwzględnie}) \textbf{ciągłą}.

\Warning W~związku z powyższym, niektórzy autorzy mówią o zmiennych losowych
ciągłych, jeśli ich dystrybuanta jest funkcją ciągłą i absolutnie (bezwględnie)
ciągłych, jeśli ich dystrybuanta jest funkcją absolutnie ciągłą.



\lecture{03.11.2008}

Z~teorii całki Lebesgue'a w~$\mathbb{R}^1$ wynika, że
\[
  f(x) = \frac{dF(x)}{dx}
\]
dla~prawie wszyskich $x$ (czyli z wyjątkiem zbioru miary Lebesgue'a równej $0$),
w~szczególności dla~każdego $x$, dla~którego $f(x)$ jest ciągła. Zauważmy
jeszcze, że jeśli zmienna losowa $X$ jest typu ciągłego, to
\[
  P(a\leq X \leq b) = F(b) - F(a) = \int_a^b f(x) \dd{x}
\]
Ponadto, jeśli $x_0$ jest punktem, dla którego istnieje funkcja gęstości $f(x)$,
to
\[
  P(X = x_o) = \int_{x_0}^{x_0} f(x) \dd{x} = 0
\]
(ciągle zakładamy, że zmienna losowa $X$ jest typu ciągłego). Mimo, że $x_0$
jest możliwą wartością zmiennej losowej $X$, czyli odpowiada mu zdarzenie
elementarne. Zatem z tego, że prawdopodobieństwo zdarzenia związanego ze zmienną
losową typu \textbf{ciągłego} równa się $0$ nie wynika, że jest to zdarzenie
niemożliwe. Zdarzenie takie uznajemy za \textbf{bardzo mało prawdopodobne}.
Analogicznie
\[
  P(X \neq x_0) = 1
\]
mimo że nie jest to zdarzenie pewne; mówimy o takim zdarzeniu, że zachodzi
ono \textbf{prawie na pewno}.

\subsection{Parametry rozkładu zmiennej losowej}
Jak pamiętamy, \textbf{pełną} charakterystykę \textbf{dowolnej} zmiennej losowej
stanowi jej rozkład prawdopodobieństwa i odpowiadająca mu jedno-jednoznacznie
dystrybuanta. Wiemy także, że w~przypadku zmiennej losowej typu skokowego jej
pełną charakterystykę stanowi również funkcja gęstości, a dla zmiennej losowej
typu ciągłego funkcja gęstości.  W wielu sytuacjach praktycznych wystarczy
znajomość pewnych prostszych parametrów rozkładu prawdopodobieństwa zmiennej
losowej.

Poniżej omówimy dwie grupy takich parametrów, mianowicie momenty i~parametry
pozycyjne.

\begin{definition}
  Wyrażenie
  \begin{eqnarray}
    E(X-c)^l &= & \int_{-\infty}^{\infty}(x-c)^l\dd{F(x)} = \\
    \nonumber
    &= & \begin{dcases}
      \sum_i(x_i-c)^lp_i & \textrm{, gdy $X$ jest typu skokowego} \\
      \int_{-\infty}^{\infty}(x-c)^lf(x)\dd{x} & \textrm{, gdy $X$ jest typu ciągłego}
    \end{dcases}
  \end{eqnarray}
  nazywamy \textbf{momentem rzędu $l$ względem punktu $c$} ($c \in \mathbb{R}$)
  zmiennej losowej $X$.
\end{definition}

Jeśli $c=0$, to mówimy o \textbf{momencie zwykłym} rzędu $l$ zmiennej
losowej~$X$. Dla~$l=1$ otrzymujemy tzw.~\textbf{wartość średnią}
(\textbf{oczekiwaną}, \textbf{nadzieję matematyczną}) zmiennej losowej $X$.

Zatem:

\begin{eqnarray}
  m_1 = m = E(X) & = & \int_{-\infty}^{\infty}x\dd{F(x)} = \\
  \nonumber
  & = & \begin{dcases}
    \sum_ix_ip_i & \textrm{, gdy $X$ jest typu skokowego}\\
    \int_{-\infty}^{\infty}xf(x)\dd{x} & \textrm{, gdy $X$ jest typu ciągłego}
  \end{dcases}
\end{eqnarray}

Bezpośrednio z definicji wartości średniej wynika następujące twierdzenie:

\begin{theorem}
  Wartość średnia (uśredniona po zbiorze) ma następujące własności:
  \begin{enumerate}
    \item[a)] $E(C) = c$, gdzie $C$ jest zmienną losową przyjmującą wartość $c$
      z~prawdopodobieństwem równym 1
    \item[b)] $E(cX) = cE(X)$
    \item[c)] $E(X+Y) = E(X) + E(Y)$
  \end{enumerate}
\end{theorem}

Ważne znaczenie mają momenty względem punktu  $c=m$, zwane momentami centralnymi
i~oznaczane przez $\mu$.

\begin{definition}
  \textbf{Moment centralny rzędu $l$, $\mu_l$,} wyraża~się wzorem:
  \begin{equation}
    \mu_l = E(X-m)^l
  \end{equation}
\end{definition}

Oczywiście $\mu_1 = 0$; natomiast $\mu_2$ nazywamy \textbf{wariancją} (albo
\textbf{dyspersją}) zmiennej losowej $X$ i~oznaczamy na~ogół przez
$\mathscr{D}^2(X)$ lub~$\sigma_{(*)}^2$.

Łatwo zauważyć, że wariancja jest miarą rozproszenia wartości zmiennej
losowej $X$ wokół jej  wartości średniej.

Pierwiastek kwadratowy z wariancji, $\mathscr{D}(X)$ lub $\sigma_{(*)}$ nazywamy
\textbf{odchyleniem standardowym (dewiacją)}. Łatwo wykazać następujące
twierdzenie:

\begin{theorem}
  Wariancja $\mathscr{D}^2(X)$ ma następujące własności:
  \begin{enumerate}
    \item[a)] $\mathscr{D}^2(X+c) = \mathscr{D}^2{(X)}$
    \item[b)] $\mathscr{D}^2(X) = E(X^2) - E^2(X)\;\;(=m_2-m^2)$
    \item[c)] $\bigwedge_{c\neq m}\mathscr{D}^2(X)<E(X-c)^2$
    \item[d)] Jeżeli wariancja jest skończona, a więc $\mathscr{D}^2(X)<\infty$,
      to dla każdego $\epsilon > 0$:
      \[
        P\left\{\omega:|X(\omega)-m|\geq \epsilon\right\}\leq
        \frac{\sigma^2}{\epsilon^2}.
      \]
      Jest to nierówność \textbf{Czebyszewa} (lub $\epsilon = t\sigma$).
  \end{enumerate}
\end{theorem}

W domu: udowodnić punkty a), b) i c) powyższego twierdzenia (obowiązuje na
egzaminie).

Zauważmy, że z własności b) wynika, że
\[
  \mathscr{D}^2(C) = 0,\;\mathscr{D}^2(cX) = c^2\mathscr{D}^2(X)
\]
Często wykorzystuje się pojęcie zmiennej losowej unormowanej.

\begin{definition}
  Mówimy, że zmienna losowa $X$ jest \textbf{unormowana (standaryzowana)}, jeśli
  \[
    E(X) = 0 \land \mathscr{D}^2(X)=1
  \]
\end{definition}

Łatwo wykazać następujące twierdzenie:

\begin{theorem}
  Dla dowolnej zmiennej losowej $X$, zmienna losowa $Y$ dana wzorem
  \[
    Y = \frac{X-E(X)}{\mathscr{D}(X)}
  \]
  jest unormowana.
\end{theorem}
\begin{proof}
  TODO
\end{proof}

Przejdźmy teraz do zdefiniowania głównych parametrów pozycyjnych.

\begin{definition}
  \textbf{Wartością modalną} zmiennej losowej $X$ nazywamy punkt, w którym maksimum osiąga:
  \begin{itemize}
    \item funkcja prawdopodobieństwa, gdy $X$ jest typu skokowego;
    \item funkcja gęstości, gdy $X$ jest typu ciągłego.
  \end{itemize}
\end{definition}

Widzimy zatem, że~zmienna losowa może mieć jedną wartość modalną, wiele takich
wartości, lub w~ogóle nie~posiadać wartości modalnej.

\begin{definition}
  Wartość $x_p$ spełniającą dla każdego $p$, $0<p<1$, nierówności:
  \[
    P(X\leq x_p)\geq p
  \]
  \[
    P(X\geq x_p)\geq 1-p
  \]
  nazywamy \textbf{kwantylem rzędu $p$} zmiennej losowej $X$.
\end{definition}

Kwantyl rzędu $0,5$ ($x_{0,5}$ lub $x_{1/2}$) nazywamy \textbf{medianą},
a~kwantyle rzędu $0,25$, $0,75$ nazywamy \textbf{kwartylami}. Zauważmy,
że~mediana istnieje dla dowolnej zmiennej losowej, podczas gdy wartość średnia
może nie istnieć (odpowiednia całka lub suma dąży do nieskończoności).

\subsection{Wielowymiarowe zmienne losowe}

\begin{definition}
  Układ $n$ funkcji rzeczywistych $(X_1,X_2,\dotsc,X_n)$ określony na~$\Omega$
  nazywamy \textbf{$n$\dywiz wymiarową (lub łączną) zmienną losową}
  (lub \textbf{wektorem losowym}) jeśli przeciwobraz każdego zbioru
  borelowskiego w~$\mathbb{R}^n$ otrzymany za pomocą tego układu jest zdarzeniem
  losowym.
\end{definition}

Inaczej mówiąc $n$-wymiarowa zmienna losowa to odwzorowanie
\[
  (X_1,X_2,\dotsc,X_n):\Omega \mapsto \mathbb{R}^n
\]
mierzalne względem miary probabilistycznej $P$.

Rozkład prawdopodobieństwa i dystrybuantę zmiennej losowej $n$\dywiz wymiarowej
definiujemy analogiczne jak dla zmienej losowej jednowymiarowej. Podajmy
definicję dystrybuanty:
\begin{definition}
  Funkcję rzeczywistą  określoną w $\mathbb{R}^n$ wzorem (przecinek oznacza
  koniunkcję!):
  \[
    F(x_1, x_2,\dotsc, x_n) = P(X_1 < x_1, X_2 < x_2,\dotsc, X_n < x_n)
  \]
  nazywamy \textbf{$n$-wymiarową (łączną) dystrybuantą} wektora
  losowego $(X_1,X_2,\dotsc,X_n)$.
\end{definition}

W dalszym ciągu zajmiemy się zmienną losową dwuwymiarową $(X,Y)$, o łącznej
dystrybuancie:
\[
  F(x,y) = P(X<x, Y<y)
\]

Łatwo zauważyć, że~dystrybuanta $F(x,y)$ jest względem każdego argumentu funkcją
nieujemną, niemalejącą i~przynajmniej lewostronnie ciągłą, oraz:
\[
  F(-\infty, y) = F(x, -\infty) = 0
\]
\[
  F(+\infty, +\infty) = 1
\]

Jednak, w~przeciwieństwie do~zmiennej losowej jednowymiarowej, powyższe własności
nie~wystarczają na~to, by funkcja $F(x,y)$ była łączną dystrybuantą zmiennej
losowej $(X,Y)$. Musi być mianowicie spełniony warunek dla dowolnych
$x_1 < x_2, y_1 < y_2$:
\begin{equation}
  F(x_2,y_2)-F(x_2, y_1) - F(x_1, y_2) + F(x_1, y_1) \geq 0
\end{equation}

Wynika to z faktu, iż prawdopodobieństwo tego, że zmienna losowa $(X,Y)$
przyjmie wartość należącą do prostokąta $(x_1\leq X\leq x_2, y_1\leq Y\leq y_2)$
musi być nieujemne.

Zdefiniujmy zmiennej losowe dwuwymiarowe typu skokowego i typu ciągłego.

\begin{definition}
  Zmienną losową $(X,Y)$ nazywamy typu skokowego (lub dyskretną), jeśli może
  przyjmować co najwyżej przeliczalną liczbę par $(x_i, y_k)$ zwanych
  \textbf{punktami skokowymi} z prawdopodobieństwami $p_{ik}$ zwanymi
  \textbf{skokami (masami)}.
\end{definition}



\lecture{17.11.2008}

Oczywiście:
\[
  \bigwedge_{i,k}p_{ik}\geq 0, \sum_k\sum_ip_{ik}=1
\]

Natomiast funkcja:
\[
  P(X=x_i, Y=y_k) = p_{ik},\;i=1,2,\cdots;\;k=1,2,\cdots
\]
definiuje \textbf{dwuwymiarową (łączną) funkcję prawdopodobieństwa} tej
zmiennej losowej. \textbf{Dwuwymiarowa (łączna) dystrybuanta} rozpatrywanej
zmiennej losowej ma natomiast postać:

\begin{equation}
  F(x,y) = \sum_{x_i<x} \sum_{y_k<y} p_{ik}
\end{equation}

\begin{definition}
  Zmienną losową $(X,Y)$ nazwiemy \textbf{typu ciągłego}, jeśli dystrybuanta
  $F(x,y)$ jest funkcją absolutnie (bezwzględnie) ciągłą, tzn.~istnieje taka
  funkcja rzeczywista $f(x,y)$, że dla każdego $(x,y)\in\mathbb{R}^2$ zachodzi:
  \begin{equation}
    F(x,y) = \int_{-\infty}^y \int_{-\infty}^x f(x, y) \dd{x}\dd{y}
  \end{equation}
  gdzie całkowanie odbywa się w sensie Lebesgue'a.
\end{definition}

Funkcja $f(x,y)$ nazywa się \textbf{dwuwymiarową (łączną) funkcją gęstości
  prawdopodobieństwa} zmiennej losowej $(X,Y)$ typu ciągłego.

Z~teorii całki Lebesgue'a w~$\mathbb{R}^n$ wynika, że~dla~prawie wszystkich
$(x,y) \in \mathbb{R}^2$ (z~wyjątkiem zbioru miary Lebesgue'a równej 0),
w~szczególności w~każdym punkcie $(x,y)$, w~którym funkcja $f(x,y)$ jest ciągła,
zachodzi:
\begin{equation}
  f(x, y) = \frac{\partial^2F(x, y)}{\partial x\partial y}
\end{equation}

Omówimy obecnie tzw.~rozkłady brzegowe. Jeżeli dla zmiennej losowej $(X,Y)$
interesuje nas rozkład tylko jednej z~jej zmiennych losowych, podczas gdy druga
może przyjmować dowolną wartość, to mówimy o~\textbf{rozkładzie brzegowym}
odpowiedniej zmiennej.

Jeżeli zmienna losowa $(X,Y)$ jest typu skokowego, to:

\begin{eqnarray}
  \bigwedge_i P(X = x_i) = P(X = x_i, Y = y_1) + P(X = x_i, Y = y_2) + \dotsc = \sum_kp_{ik} = p_{i\bullet} \\
  \bigwedge_k P(Y = y_k) = P(X = x_1, Y = y_k) + P(X = x_2, Y = y_k) + \dotsc = \sum_i p_{ik} = p_{\bullet k}
\end{eqnarray}

Powyższe wzory definiują funkcję prawdopodobieństwa rozkładu brzegowego
odpowiednio zmiennych losowych $X$ i~$Y$ zmiennej losowej typu skokowego $(X,Y)$.

Jeżeli natomiast zmienna losowa $(X,Y)$ jest typu ciągłego, to~funkcja gęstości
rozkładu brzegowego zmiennej losowej $X$ wyraża się wzorem:
\[
  f_1(x) = \int_{-\infty}^{\infty} f(x, y) \dd{y}.
\]

Natomiast funkcja gęstości rozkładu brzegowego zmiennej losowej $Y$ wyraża~się
wzorem:
\[
  f_2(y) = \int_{-\infty}^{\infty} f(x, y) \dd{x}
\]

Ogólnie dla zmiennej losowej $(X,Y)$ dystrybuanta rozkładu brzegowego zmiennej
losowej $X$ dana jest wzorem:
\[
  F_1(x) = F(x, \infty) = P(X < x, Y < \infty)
\]
natomiast dystrybuanta rozkładu brzegowego zmiennej losowej $Y$ wyraża się
wzorem:
\[
  F_2(y) = F(\infty, y) = P(X < \infty, Y < y)
\]

W szczególności, jeśli zmienna losowa $(X,Y)$ jest typu skokowego, to np.:
\[
  F_1(x) = \sum_{x_i<x} p_{i\bullet} = \sum_{x_i<x} \sum_k p_{ik}
\]

Jesli natomiast zmienna losowa $(X,Y)$ jest typu ciągłego, to np.:
\[
  F_1(x) = \int_{-\infty}^x f_1(x) \dd{x} = \int_{-\infty}^x \int_{-\infty}^{\infty} f(x,y) \dd{y}\dd{x}
\]

Ważnym zagadnieniem są \textbf{rozkłady warunkowe}, np.~zmiennej losowej $Y$
pod warunkiem, że zmienna losowa $X$ przyjmie ustaloną wartość.

Jeśli zmienna losowa $(X,Y)$ jest typu skokowego, to \textbf{dystrybuanta
  warunkowa} zmiennej losowej $Y$, pod warunkiem, że $X=x_i$, wyraża~się wzorem:
\begin{equation}
  F(y|x_i) = \sum_{y_k < y} P(y_k|x_i),
\end{equation}
gdzie $P(y_k|x_i)$ jest \textbf{warunkową funkcją prawdopodobieństwa}
i~wyraża~się wzorem:
\begin{eqnarray}
  P(y_k|x_i) &=& P(Y = y_k|X = x_i) = \frac{P(Y = y_k,X = x_i)}{P(X = x_i)} = \\
  \nonumber
  &=& \frac{p_{ik}}{p_{i\bullet}},\;k=1,2,\dotsc
\end{eqnarray}

Jeśli natomiast zmienna losowa $(X,Y)$ jest typu ciągłego, to, przy pewnych
założeniach dotyczących granicy, dystrybuanta warunkowa zmiennej losowej $Y$
pod warunkiem, że $X=x$ wyraża się wzorem:
\begin{equation}
  F(y|x)=\int_{-\infty}^yf(y|x)\dd{y},
\end{equation}
gdzie $f(y|x)$ jest \textbf{warunkową funkcją gęstości} zmiennej losowej $Y$,
wyrażającą się wzorem:
\begin{equation}
  f(y|x) = \frac{f(x,y)}{f_1(x)},~f_1(x) > 0
\end{equation}
Widzimy zatem, że:
\[
  F(y|x)f_1(x) = \int_{-\infty}^y f(x,y)\d{y}
\]
czyli:
\begin{eqnarray}
  \nonumber
  \int_{-\infty}^{\infty} F(y|x) f_1(x) \dd{x} & = & \int_{-\infty}^{\infty} \int_{-\infty}^y f(x,y) \dd{y}\dd{x} \\
  \nonumber
  & = & \int_{-\infty}^y \int_{-\infty}^{\infty} f(x,y) \dd{x}\dd{y} = F_2(y)
\end{eqnarray}
Powyższy wzór stanowi uogólnienie twierdzenia o prawdopodobieństwie całkowitym
na zmienne losowe typu ciągłego, czyli na nieprzeliczalną liczbę zdarze
losowych.

Mając dane rozkłady warunkowe, możemy zdefiniować \textbf{momenty warunkowe};
przykładowo \textbf{warunkową wartość średnią} zmiennej losowej~$Y$,
pod warunkiem, że~$X=x$ (lub~$X=x_i$ dla~zmiennej losowej typu skokowego;
jest~to zapis ogólny) definiujemy ogólnie jako całkę Stieltjesa:
\begin{eqnarray}
  E(Y|X = x) &=& \int_{-\infty}^{\infty} y \dd{F(y|x)} = \\
  \nonumber
  &=& \begin{dcases}
    \sum_ky_kP(y_k|x_i) & \textrm{gdy $(X,Y)$ jest typu skokowego} \\
    \int_{-\infty}^{\infty}yf(y|x)\dd{y} & \textrm{gdy $(X,Y)$ jest typu ciągłego}
  \end{dcases}
\end{eqnarray}
(zakładamy, że $f_1(x)>0$). Zauważmy, że słuszny jest wzór:
\begin{eqnarray}
  E(Y) & = & \int_{-\infty}^{\infty}E(Y|X=x)\dd{F_1(x)} = \\
  \nonumber
  & = & \begin{dcases}
    \sum_iE(Y|X=x_i)P(X=x_i) & \textrm{, gdy $(X,Y)$ jest typu skokowego} \\
    \int_{-\infty}^{\infty}E(Y|X=x)f_1(x)\dd{x} & \textrm{, gdy $(X,Y)$ jest typu ciągłego}
  \end{dcases}
\end{eqnarray}
Ogólnie, jeśli chcemy wyliczyć jakiś moment rozkładu zmiennej losowej np.~$Y$,
która zależy od zmiennej losowej $X$, to postępujemy dwuetapowo:
\begin{enumerate}
  \item[\textbf{1)}] wyznaczamy ten moment, pod warunkiem, że zmienna losowa $X$
    przyjmie określoną wartość (czyli liczymy $E(Y|X = x)$),
  \item[\textbf{2)}] uśredniamy otrzymany wynik po rozkładzie brzegowym zmiennej
    losowej $X$, czyli wyliczamy całkę Stieltjesa z tego wyniku względem
    dystrybuanty $F_1(x)$.
\end{enumerate}

Przejdźmy obecnie do pojęć związanych z niezależnością zmiennych losowych.
\begin{definition}
  Mówimy, że~zmienne losowe $X$ i~$Y$ są~\textbf{niezależne}, jeśli dla
  dowolnych zbiorów borelowskich $S_1,S_2$ odpowiednio na osi $x$ i~$y$,
  zdarzenia $Z_1=\{\omega:X(\omega)\in S_1\}, Z_2=\{\omega:Y(\omega)\in S_2\}$
  spełniają relację:
  \begin{equation}
    P(Z_1\cap Z_2) = P(Z_1)\cdot P(Z_2)
  \end{equation}
  czyli są~\textbf{zdarzeniami niezależnymi}.
\end{definition}

Łatwo zauważyć, że~powyższa relacja zachodzi wtw, gdy~dla~każdego
$(x,y)\in\mathbb{R}^2$ prawdziwy jest warunek:
\[
  F(x,y)=F_1(x)\cdot F_2(y)
\]
W~związku z~powyższym, niektórzy autorzy podają ten warunek jako definicję
niezależności zmiennych losowych $X$ i~$Y$.

Oczywiście warunek ten dla zmiennych losowych typu skokowego można również zapisać w~postaci:
\[
  \bigwedge_{i,k}p_{ik}=p_{i\cdot}\cdot p_{\cdot k}
\]
Natomiast dla zmiennych losowych typu ciągłego w postaci:
\[
  f(x,y)=f_1(x)\cdot f_2(y)
\]

Zauważmy również, że jeśli $X$ i $Y$ są niezależne, to:
\[
  F(x|y) = F_1(x),~F(y|x)=F_2(y)
\]

Uogólnimy obecnie definicję niezależności zmiennej losowej dla $n>2$.
\begin{definition}
  Mówimy, że zmienne losowe $X_1,X_2,\dotsc,X_n$ są \textbf{niezależne}, jeśli
  dla~dowolnych zbiorów borelowskich $S_1,S_2,\dotsc,S_n$ odpowiednio na osi
  $x_1,x_2,\dotsc,x_n$, zdarzenia $Z_i=\{\omega:X_i(\omega)\in S_i\},~i=1,2,
  \dotsc,n$ spełniają relację:
  \begin{equation}
    P(Z_1\cap Z_2 \cap\dotsb\cap Z_n) = P(Z_1)\cdot P(Z_2) \dotsm P(Z_n)
  \end{equation}
\end{definition}

Łatwo zauważyć, że powyższa relacja zachodzi wtw, gdy~dla~dowolnego
$(x_1,x_2,\dotsc,x_n)\in\mathbb{R}^n$ zachodzi równość:
\[
  F(x_1,x_2,\dotsc,x_n) = F(x_1) \cdot F(x_2) \dotsm F(x_n)
\]
\begin{definition}
  Mówimy, że ciąg zmiennych losowych $X_1,X_2,\dotsc$ jest \textbf{ciągiem
    zmiennych losowych niezależnych}, jesli dla~dowolnego $n$ zmienne losowe
  $X_1,X_2,\dotsc,X_n$ są niezależne.
\end{definition}



\lecture{24.11.2008}

Wykażemy teraz dwa ważne twierdzenia.

\begin{theorem} \label{thm:two:mean}
  Wartość średnia iloczynu dowolnej skończonej liczby niezależnych zmiennych
  losowych o~skończonych wartościach średnich równa się iloczynowi wartości
  średnich tych zmiennych.
\end{theorem}

\begin{proof}
  Dla $n=2$ (dla $n>2$ dowód zachodzi przez indukcję).
  Niech $X, Y$ będą niezależnymi zmiennymi losowymi typu skokowego. Wówczas:
  \begin{eqnarray}
    \nonumber
    E(X Y) &=& \sum_i\sum_k p_{ik}x_iy_k = \sum_i\sum_k x_iy_kp_{i\cdot}p_{\cdot k} =\\
    \nonumber
    &=& \sum_ix_ip_{i\cdot}\sum_ky_kp_{\cdot k} = E(X)\cdot E(Y)
  \end{eqnarray}
  Zakładając, że zmienne losowe $X, Y$ są typu ciągłego, otrzymujemy:
  \begin{eqnarray}
    \nonumber
    E(X Y) &=& \int_{-\infty}^\infty\int_{-\infty}^\infty xyf(x,y)dxdy = \int_{-\infty}^\infty\int_{-\infty}^\infty xyf_1(x)f_2(y)dxdy = \\
    \nonumber
    &=& \int_{-\infty}^\infty xf_1(x)dx\cdot\int_{-\infty}^\infty yf_2(y)dy = E(X)\cdot E(Y)
  \end{eqnarray}
\end{proof}

\begin{theorem}
  Wariancja sumy dowolnej skończonej liczby niezależnych zmiennych losowych
  o~skończonych wariancjach jest równa sumie wariancji tych zmiennych.
\end{theorem}

\begin{proof}
  (TODO dla $n=2$; dla $n>2$ dowód zachodzi przez indukcję)
\end{proof}

Zdefiniujemy obecnie podstawowe momenty zmiennej losowej $(X, Y)$.
\begin{definition}
  \textbf{Momentem zwykłym rzędu} $l+n$ zmiennej losowej $(X, Y)$ nazywamy
  wyrażenie:
  \begin{eqnarray}
    m_{ln} &=& E(X^lY^n) = \int_{-\infty}^\infty\int_{-\infty}^\infty x^ly^ndF(x,y) = \\
    \nonumber
    &=& \begin{dcases}
      \sum_i\sum_kx_i^ly_k^np_{ik} & \textrm{gdy $(X,Y)$ jest typu skokowego} \\
      \int_{-\infty}^\infty\int_{-\infty}^\infty x^ly^nf(x,y)\dd{x}\dd{y}& \textrm{gdy $(X,Y)$ jest typu ciągłego}
    \end{dcases}
  \end{eqnarray}
\end{definition}

Ogólnie mogą istnieć dwa momenty rzędu pierwszego, tu:
\begin{align*}
  m_{10} = E(X) \\
  m_{01} = E(Y) \\
  \shortintertext{trzy momenty rzędu drugiego, tu:}
  m_{20} = E(X^2) \\
  m_{11} = E(XY) \\
  m_{02} = E(Y^2) \\
  \shortintertext{itd.}
\end{align*}

\begin{definition}
  \textbf{Momentem centralnym rzędu} $l+n$ zmiennej losowej $(X,Y)$ nazywamy
  wyrażenie:
  \begin{equation}
    \mu_{ln} = E[(X-m_{10})^l(Y-m_{01})^n]
  \end{equation}
\end{definition}

\begin{subequations}
  Łatwo zauważyć, że:
  \begin{align}
    \mu_{10} &= E(X-m_{10}) = 0 = \mu_{01} = E(Y-m_{01}) \\
    \mu_{20} &= E(X-m_{10})^2 = \sigma_X^2 \\
    \mu_{02} &= E(Y-m_{01})^2 = \sigma_Y^2
  \end{align}
\end{subequations}

Natomiast moment centralny $\mu_{11}$ nazywamy \textbf{kowariancją} zmiennej
losowej $(X,Y)$ i~oznaczamy również przez $k_{XY}$ lub~$cov(X,Y)$. Zatem:
\begin{equation}
  \mu_{11} = k_{XY} = cov(X,Y) = E[(X-m_{10})(Y-m_{01})]
\end{equation}

Z~twierdzenia~\ref{thm:two:mean}, jeśli zmienne losowe $X$ i~$Y$ są~niezależne,
to $\mu_{11} = 0$. Twierdzenie odwrotne nie~jest prawdziwe.

\begin{definition}
  Jeśli dla zmiennych losowych $X, Y$ istnieją $\sigma_X$ i $\sigma_Y$,
  to wyrażenie:
  \begin{equation}
    \rho_{XY} = \frac{k_{XY}}{\sigma_X\sigma_Y}
  \end{equation}
  nazywamy \textbf{współczynnikiem korelacji} zmiennych losowych $X$ i~$Y$.
\end{definition}

Łatwo wykazać, że $-1\leq \rho_{XY} \leq 1$. Jeśli dla zmiennych losowych $X$
i~$Y$ $\rho_{XY} = 0$, to mówimy, że zmienne te są~\textbf{nieskorelowane}.
Z~twierdzenia~\ref{thm:two:mean} widzimy, że~jeśli zmienne losowe są~niezależne,
to są~również nieskorelowane, ale~nie~odwrotnie. Definicję momentu zmiennej
losowej dwuwymiarowej uogólnia~się w~naturalny sposób na~$n>2$, np.~dla~wektora
losowego $(X_1, X_2,\dotsc, X_n)$ kowariancję par różnych zmiennych losowych
$X_i,X_j$ definiujemy jako:
\begin{equation}
  k_{ij}=E[(X_i-m_i)(X_j-m_j)],\;i,j=1,2,\dotsc,n, i\neq j
\end{equation}
gdzie $m_i = E(X_i), m_j = E(X_j)$.

Dla $i=j$ otrzymujemy oczywiście wariancję poszczególnych zmiennych.
\begin{definition}
  Macierz
  \begin{equation}
    K = \begin{bmatrix}
      k_{11} & k_{12} & \ldots &k_{1n} \\
      k_{21} & k_{22} & \ldots &k_{2n} \\
      \vdots & \vdots & \ddots \\
      k_{n1} & k_{n2} & \ldots &k_{nn} \\
    \end{bmatrix}
  \end{equation}
  nazywamy \textbf{macierzą korelacyjną} wektora losowego $(X_1,X_2,\ldots,X_n)$
  (oczywiście macierz ta jest symetryczna, $k_{ij}=k_{ji}$).
\end{definition}

\subsection{Funkcje charakterystyczne}
Jednym z ważnych problemów jest sumowanie niezależnych zmiennych losowych,
o którym będziemy mówili w punkcie 8 (twierdzenia graniczne).  Wyznaczenie
rozkładu prawdopodobieństwa sumy zmiennych losowych, także niezależnych,
jest rachunkowo skomplikowane; żeby je uprościć opracowano podejścia, które
w~ogólności polegają na~przyporządkowaniu rozpatrywanym zmiennym losowym
(czyli ich rozkładom prawdopodobieństwa) wzajemnie jednoznacznych pewnych
funkcji zmiennej rzeczywistej, które mają tę własność, że rozkładowi
prawdopodobieństwa sumy niezależnych zmiennych losowych odpowiada iloczyn tych
funkcji. Klasy funkcji o tej własności to:
\begin{itemize}
  \item funkcje tworzące prawdopodobieństwa
  \item funkcje tworzące momenty
  \item funkcje charakterystyczne
\end{itemize}
W dalszym ciągu omówimy tylko te ostatnie jako najogólniejsze.

\begin{definition}
  \textbf{Funkcją charakterystyczną} zmiennej losowej $X$ o dystrybuancie $F(x)$
  nazywamy funkcję zespoloną zmiennej rzeczywistej $t$, daną wzorem:
  \begin{equation}
    \phi_t = E(e^{itX}) = \int_{-\infty}^{\infty}e^{itx}dF(x)
  \end{equation}
\end{definition}

Jako, że
\[
  \bigwedge_{x,t}|e^{itx}| = 1
\]
stąd powyższa całka istnieje dla dowolnej zmiennej losowej X.

Łatwo wykazać następujące twierdzenie:

\begin{theorem}
  Jeśli $X, Y$ są niezależnymi zmiennymi losowymi o dystrybuantach $F(x), F(y)$,
  to:
  \begin{equation}
  \phi_{X+Y}(t) = \phi_X(t)\cdot\phi_Y(t)
  \end{equation}
\end{theorem}

\begin{proof}
  TODO
\end{proof}

Można wykazać, że zachodzi następujący związek pomiędzy funkcją
charakterystyczną zmiennej losowej $X$ a jej momentami:
\begin{equation}
  \left.\frac{d^k\phi(t)}{dt^k}\right|_{t=0}=i^km_k,\;k=1,2,\dotsc
\end{equation}
Definicję funkcji charakterystycznej uogólnia się w sposób naturalny na $n>2$,
manowicie dla $(X_1, X_2, \dotsc, X_n)$:
\begin{equation}
  \phi(t_1,t_2, \dotsc, t_n) = E[e^{i(t_1X_1+t_2X_2+\dotsb+t_nX_n)}]
\end{equation}


\subsection{Przykładowe rozkłady prawdopodobieństwa}
\subsubsection{Rozkład jednopunktowy}
\begin{definition}
  Mówimy, że zmienna losowa $X$ ma \textbf{rozkład jednopunktowy} jeśli istnieje
  taki punkt $x_0$, że:
  \begin{equation}
    P(X=x_0) = 1
  \end{equation}
\end{definition}
Widzimy zatem, że cała masa zmiennej losowej o rozkładzie jednopunktowym
skupiona jest w punkcie $x_0$. Oczywiście dystrybuanta takiej zmiennej losowej
wyraża się wzorem:
\begin{equation}
  F(x) = \begin{cases}
    0 & \text{dla } x \leq x_0 \\
    1 & \text{dla } x > x_0
  \end{cases}
\end{equation}
Oczywiste jest również, że $E(X)=x_0$ oraz $\mathscr{D}^2(X)=0$.


\subsubsection{Rozkład dwupunktowy}
\begin{definition}
  Mówimy, że zmienna losowa $X$ ma \textbf{rozkład dwupunktowy}, jeśli istnieją
  takie punkty $x_1, x_2$, że:
  \begin{equation}
    P(X=x_1)=p,\;P(X=x_2)=1-p,\;\;(0<p<1)
  \end{equation}
\end{definition}

Oczywiście dystrybuanta takiej zmiennej losowej wyraża się wzorem:
\begin{eqnarray}
  F(x) = \begin{cases}
    0 & \text{dla } x \leq x_1 \\
    p & \text{dla } x_1 < x \leq x_2 \\
    1 & \text{dla } x > x_2
  \end{cases}
\end{eqnarray}

\begin{align*}
  E(X) &= px_1 + (1 - p)x_2 \\
  \mathscr{D}^2(X) &= p(1 - p)(x_1 - x_2)^2
\end{align*}

\subsubsection{Rozkład dwumianowy (Bernoulliego)\footnote{U większości autorów
  angielskich rozkładem Bernoulliego określa się rozkład zero-jedynkowy,
  tj.~rozkład dwupunktowy o nośniku $\{0,1\}$.}}
Zmienną losową o~rozkładzie dwumianowym otrzymujemy w~następującym schemacie
doświadczeń, zwanym \textbf{schematem Bernoulliego}. Powtarzamy $n$-krotnie
pewne doświadczenie losowe. W~wyniku każdego powtórzenia może wystąpić zdarzenie
$A$ z prawdopodobieństwem $p$ lub zdarzenie $A'$ z~prawdopodobieństwem $q=1-p$.
Zakładamy, że \textbf{wyniki doświadczeń są niezależne}, czyli~że
prawdopodobieństwo zdarzenia $A$ jest takie samo w każdym powtórzeniu
(nie zależy od tego, które to jest powtórzenie ani od tego, ile razy lub
w jakiej kolejności wystąpiło to zdarzenie w poprzednich powtórzeniach). Zatem
zmienna losowa $X$ odpowiadająca temu schematowi doświadczeń może przyjmować
wartości $0,1,2,\ldots,n$, przy czym na to, aby $X=k$ potrzeba i wystarcza,
by zdarzenie $A$ wystąpiło w $k$-powtórzeniach ($A'$ w $n-k$ powtórzeniach).
Ponieważ powtórzenia są niezależne, więc:
\[
  P(\underbrace{A\cap A\cap\ldots\cap A}_{k\textrm{~razy}}\cap
  \underbrace{A'\cap A'\cap\ldots\cap A' }_{n-k\textrm{~razy}}) = p^kq^{n-k}
\]
Oczywiście dla dowolnego $k$ zdarzenie polegające na $k$-krotnym wystąpieniu
$k$ w~$n$ powtórzeniach może nastąpić $n \choose k$ sposobami,
czyli ostatecznie:
\begin{equation} \label{eq:binom}
  P(X=k) = {n\choose k} p^kq^{n-k},\;k=0,1,\dotsc,n
\end{equation}

\begin{definition}
  Zmienną losową $X$ przyjmującą wartości $k=0,1,\dotsc,n$ zgodnie z~funkcją
  prawdopodobieństwa~\ref{eq:binom} nazywamy \textbf{zmienną losową o~rozkładzie
    dwumianowym (Bernoulliego)}.
\end{definition}

W domu: wyznaczyć $E(X)$ i $\mathscr{D}^2(X)$ (skorzystać ze wzoru dwumianowego Newtona).
\lecture{01.12.2008}
\subsubsection{Rozkład Poissona}
\begin{definition}
Mówimy, że zmienna $X$ ma \textbf{rozkład Poissona}, jeśli jej funkcja prawdopodobieństwa dana jest wzorem:
\begin{equation}
P(X=k) = \frac{\lambda^k}{k!} e^{-\lambda}, ~~\lambda>0,~ k=0,1,2,\ldots
\end{equation}
\end{definition}
Wyliczmy wartość średnią i wariancję zmiennej losowej $X$. Wartość średnia:
\[ E(X) = \sum_{k=1}^\infty k\frac{\lambda^k}{k!} e^{-\lambda} = e^{-\lambda} \sum_{k=1}^\infty \frac{\lambda^k}{(k-1)!} = 
 \lambda e^{-\lambda}\cdot \underbrace{\sum_{k=1}^\infty \frac{\lambda^{k-1}}{(k-1)!}}_{e^\lambda \textrm{~~z Maclaurina}} = \lambda
\]
Wariancję obliczymy ze wzoru $\mathscr{D}^2(X)=E(X^2)-E^2(X)$:
\[ E(X^2) = \sum_{k=1}^\infty k^2 \frac{\lambda^k}{k!}e^{-\lambda} = \sum_{k=2}^\infty k(k-1)\frac{\lambda^k}{k!}e^{-\lambda} + 
\underbrace{\sum_{k=1}^\infty k\frac{\lambda^k}{k!}e^{-\lambda}}_{\lambda} = \lambda^2e^{-\lambda} \underbrace{ \sum_{k=2}^\infty
 \frac{\lambda^{k-2}}{(k-2)!}}_{e^\lambda} + \lambda = \lambda^2+\lambda \]
Zatem:
\[ \mathscr{D}^2(X) = \lambda^2 + \lambda - \lambda^2 = \lambda \]
Związek pomiędzy rozkładem Bernoulliego a rozkładem Poissona podaje następujące twierdzenie:
\begin{theorem}
Jeśi zmienne losowe $X_n$, n=1,2,\ldots, mają rozkład Bernoulliego o funkcji prawdopodobieństwa $P(X=k) = {n\choose k} p^kq^{n-k},~~k=0,1,\ldots,n$, oraz
$np=\lambda, ~\lambda>0,~n=1,2,\ldots$, to:
\begin{equation}
\lim_{n\rightarrow\infty} P(X_n=k)=\frac{\lambda^k}{k!}e^{-\lambda}
\end{equation}
\end{theorem}
Z twierdzenia tego wynika przede wszystkim to, że rozkład Poissona ma tę samą naturę z punktu widzenia niezależności, co rozkład Bernoulliego, 
a ponadto rozkład Bernoulliego można zastąpić rozkładem Poissona dla dużych $n$ (w praktyce $n>20$) i małych $p=\frac{\lambda}{n}$ (w praktyce $p<0.2$).

Powyżej rozpatrywaliśmy przykłady rozkładów zmiennych losowych typu skokowego. W kolejnych 3 punktach podamy ważne przykłady rozkładów zmiennych losowych typu ciągłego.

\subsubsection{Rozkład równomierny}
\begin{definition}
Mówimy, że zmienna losowa $X$ ma rozkład równomierny na skończonym przedziale $[a,b]$ jeśli jej funkcja gęstości dana wzorem:
\begin{eqnarray}
f(x) = \left\{ \begin{array}{ll}
\frac{1}{b-a} & \textrm{dla $x \in [a,b]$} \\
0 & \textrm{dla $x\notin [a,b]$}
\end{array} \right.
\end{eqnarray}
\end{definition}
W domu: wylicz $E(X)$ i $\mathscr{D}^2(X)$

\subsubsection{Rozkład normalny (Gaussa, gaussowski)}
\begin{definition}
Mówimy, że zmienna losowa $X$ ma \textbf{rozkład normalny (Gaussa)}, jeśli jej funkcja gęstości dana jest wzorem:
\begin{equation}
f(x) = \frac{1}{b\sqrt{2\pi}}\exp \left[\frac{-(x-a)^2}{2b^2}\right] = \frac{1}{b\sqrt{2\pi}}e^\frac{-(x-a)^2}{2b^2},~~a\in (-\infty,\infty),~b>0
\end{equation}
\end{definition}
Badając tę funkcję standardowymi metodami analizy matematycznej, łatwo stwierdzamy, że $f(x)$:
\begin{itemize}
\item ma maksimum dla $x=a$
\item jest symetryczna względem prostej $x=a$
\item ma punkty przegięcia dla $x=a+b$ i $x=a-b$
\item ma asymptotę poziomą w postaci osi $X$ dla $x\rightarrow \pm\infty$
\end{itemize}
Wyliczmy wartość średnią zmiennej losowej $X$:
\[ E(X) = \int_{-\infty}^\infty x f(x)dx \]
Podstawiając $y=\frac{x-a}{b}$ mamy:
\[ x=by+a, ~~dx=b \cdot dy \]
Zatem:
\[ E(X) = \frac{1}{b\sqrt{2\pi}} \underbrace{\int_{-\infty}^{\infty}by e^\frac{-y^2}{2}b dy}_{0} +
\frac{1}{b\sqrt{2\pi}} \int_{-\infty}^{\infty}a e^\frac{-y^2}{2}b dy
= \frac{a}{\sqrt{2\pi}} \underbrace{\int_{-\infty}^\infty e^\frac{-y^2}{2}dy}_{\sqrt{2\pi}} = a \]
Stosując to samo podstawienie, następnie całkując przez części można wyliczyć, że:
\[ \mathscr{D}^2(X) = \sigma^2 = b^2 \]
Zatem funkcja gęstości rozkładu normalnego zapisuje się w postaci:
\[ f(x) = \frac{1}{\sigma\sqrt{2\pi}}\exp\left[-\frac{(x-m)^2}{2\sigma^2}\right] \]
lub krótko:
\[ N(m,~\sigma)~~\textrm{-- rozkład normalny o parametrach}~m~\textrm{i}~\sigma \]
Wykres funkcji gęstości rozkładu normalnego ma postać: 
\begin{center}
\textbf{TODO}: wstawić obrazek ;-)
\end{center}
Ze względu na rozpowszechnienie rozkładu normalnego oraz na fakt, że całki z jej funkcji gęstości nie można wyliczyć korzystając z całki jej funkcji elementarnych, 
opracowano tablicę rozkładu normalnego, podano w nich wartości dystrybuanty $F(x)$ zmiennej losowej normalnej 
i~unormowanej $N(0,1)$, czyli zmiennej losowej o funkcji gęstości:
\[ f(x) = \frac{1}{\sqrt{2\pi}}\exp\left(-\frac{x^2}{2}\right) \]
przy czym są to wartości wyłącznie dla $x>0$, czyli wartości:
\[ F(x) = \frac{1}{\sqrt{2\pi}}\int_{-\infty}^{x>0}e^\frac{-x^2}{2}dx \]
Żeby zatem skorzystać z tablic rozkładu normalnego musimy zmienną losową o rozkładzie normalnym unormować, a następnie doprowadzić do całek w powyższej postaci, korzystając z symetrii funkcji $f(x)$ oraz z tego, że:
\[ \frac{1}{2\pi}\int_{-\infty}^\infty e^\frac{-x^2}{2}dx = 1 \]
\begin{center}
\textbf{TODO}: wykresik w tym miejscu ;-)
\end{center}
\textbf{Przykład}.

Zmienna losowa $X$ ma rozkład $N(1,~2)$. Znaleźć prawdopodobieństwo tego, że $|X|>3$. \\
W celu skorzystania  z tablic rozkładu normalnego unormujemy najpierw zmienną losową $X$ podstawiając $Y=\frac{X-1}{2}$, zatem:
\begin{eqnarray} \nonumber
P(|X|>3) &=&P(|2Y+1|>3)=P((2Y+1)>3)+P((-2Y-1)>3)=\\\nonumber
&=&P(Y>1)+P(Y<-2)=\frac{1}{\sqrt{2\pi}}\int_1^\infty e^\frac{-y^2}{2}dy + \\\nonumber
&+&\frac{1}{\sqrt{2\pi}}\int_{-\infty}^{-2} e^\frac{-y^2}{2}dy = \textrm{(dokończ)}
\end{eqnarray}
Rozkład normalny ma wiele ważnych własności, m.in.:
\begin{itemize}
\item \textbf{jest niezmiennikiem przekształceń liniowych}, tzn. jeśli $X$ jest zmienną losową o rozkładzie normalnym, to również zmienna losowa $Y=aX+b$ ma rozkład normalny;
\item \textbf{jest nieskończenie podzielny}, tzn. jeśli $\phi(t)$ jest funkcją charakterystyczną zmiennej losowej o rozkładzie normalnym, to również
pierwiastek postaci $\sqrt[n]{\phi(t)}$ dla dowolnego $n$ jest funkcją charakterystyczną zmiennej losowej o rozkładzie normalnym;
\item \textbf{jest w pełni scharakteryzowany przez $m$ i $\sigma$}.
\end{itemize}
Dla zmiennych losowych o rozkładzie normalnym wykorzystywana jest również w praktyce nierówność 3\dywiz sigmowa ($3-\sigma$\dywiz owa); mówi ona, że jeśli zmienna losowa $X$ ma rozkład normalny, to:
\[ P(|X-m|\geq 3\sigma)\thickapprox 0.0027\]
lub
\[ P(|X-m|\leq 3\sigma)\thickapprox 0.9973 \]
Z nierówności tej wynika zatem, że rozkład normalny (rozkłady zbliżone do normalnego) są mocno skupione wokół wartości średnich.

\subsubsection{Rozkład wykładniczy}
\begin{definition}
Mówimy, że zmienna losowa $X$ ma rozkład wykładniczy, jesli jej funkcja gęstości dana jest wzorem:
\begin{eqnarray}
f(x) = \left\{\begin{array}{ll}
0 & \textrm{~~dla $x\leq0$}\\\nonumber
\lambda e^{-\lambda x} & \textrm{~~dla $x>0,~\lambda>0$}
\end{array}\right.
\end{eqnarray}
\end{definition}
W domu: wylicz $E(X)$ i $\mathscr{D}^2(X)$. \\\\
$\circledcirc$ Podamy obecnie dwie definicje dotyczące regresji:
\begin{definition}
Zbiór warunkowych wartości średnich $E(Y|X=x)$ rozpatrywany jako funkcja $x$ nazywamy \textbf{krzywą (funkcją) regresji} zmiennej losowej $Y$
względem zmiennej losowej $X$ (lub odwrotnie). 
\end{definition}
Zauważmy, że jeśli zmienne losowe $X, Y$ są niezależne, to funkcja regresji zmiennej losowej $Y$ 
względem zmiennej losowej $X$ jest prostą równoległą do osi $x$, a funkcja regresji zmiennej losowej $X$ względem zmiennej losowej $Y$ jest 
prostą równoległą do osi $y$.

W ogólności jednak  krzywe regresji nie są prostymi, w praktyce natomiast chodzi o znalezienie takiej prostej, która spośród wszystkich prostych
na płaszczyźnie $(X,Y)$ zapewnia minimalne średnie odchylenie kwadratowe wartości zmiennej losowej $Y$ od tej prostej.
\begin{center}
\textbf{TODO}: ładny wykresik z krzywą regresji ;-)
\end{center}
Jeśli zatem równanie poszukiwanej prostej to $y=ax+b$, musi być spełniony warunek:
\[ E\left\{\left[Y-(aX+b)\right]^2\right\}=min. \]
\begin{definition}
Prostą o powyższej własności nazywamy \textbf{prostą regresji drugiego rodzaju} zmiennej losowej $Y$ względem zmiennej losowej $X$ (lub odwrotnie).
\end{definition}
Uwaga: niekiedy regresją zdefiniowaną w pierwszej definicji nazywa się \textbf{regresją pierwszego rodzaju}.\\
W domu: 

wyliczyć parametry $a,~b$ prostej regresji drugiego rodzaju.
\lecture{08.12.2008}
\subsection{Ciągi zmiennych losowych i rodzaje ich zbieżności}
Poniżej zdefiniujemy główne typy zbieżnosci ciągów zmiennych losowych.
\begin{definition}
Ciąg zmiennych losowych $X_1, X_2, \ldots$ jest \textbf{zbieżny z prawdopodobieństwem 1} (``prawie na pewno'', ogólnie: ``prawie wszędzie'') do zmiennej
losowej $X$, jeśli:
\begin{equation}
P(\omega: \lim_{n\rightarrow\infty}X_n(\omega)=X(\omega))=1
\end{equation}
lub krótko:
\begin{equation}
P(\lim_{n\rightarrow\infty}X_n=X)=1
\end{equation}
\end{definition}
\begin{definition}
Ciąg zmiennych losowych $X_1, X_2, \ldots$ jest \textbf{zbieżny według prawdopodobieństwa} (stochastycznie, ogólnie: według miary) do zmiennej
losowej $X$, jeśli:
\begin{equation}
\bigwedge_{\epsilon>0} \lim_{n\rightarrow\infty} P(|X_n-X|\geq\epsilon) = 0
\end{equation}
\end{definition}
\begin{definition}
Ciąg zmiennych losowych $X_1, X_2, \ldots$ o dystrybuantach $F_1(x), F_2(x), \ldots$ jest \textbf{zbieżny według dystrybuanty} do zmiennej 
losowej $X$ o dystrybuancie $F(x)$ jeśli ciąg funkcji $F_1(x), F_2(x), \ldots$ jest zbieżny do funkcji $F(x)$.
\end{definition}
\begin{definition}
Ciąg zmiennych losowych $X_1, X_2, \ldots$ takich, że $E(X_i^2)<\infty, i=1,2,\ldots$ jest \textbf{zbieżny średnio z kwadratem} do zmiennej losowej $X$
takiej, że $E(X^2)<\infty)$, jeśli:
\begin{equation}
\lim_{n\rightarrow\infty}E(X_n-X)^2=0
\end{equation}
\end{definition}
Można wykazać, że pomiędzy powyższymi rodzajami zbieżności istnieją następujące związki:
\begin{center}
\textbf{TODO}: flowchart
\end{center}

\subsection{Twierdzenia graniczne}
Dotyczą one granicznych rozkładów ciągów niezależnych zmiennych losowych, gdy ich liczba dąży do nieskończoności. Szczególnie ważną rolę odgrywają
dwie grupy tych twierdzeń:
\begin{itemize}
 \item prawa wielkich liczb
\item centralne twierdzenia.
\end{itemize}
Prawa wielkich liczb dotyczą zachowania się ciągów średnich arytmetycznych niezależnych wyników obserwacji, gdy ich liczba dąży do nieskończoności.
Chociaż każdy wynik jest realizacją pewnej zmiennej losowej, to jednak przy rozpatrywaniu ich dużej liczby przypadkowe wpływy kompensują się wzajemnie
i~w~rezultacie przy dość ogólnych założeniach średnia arytmetyczna wyników obserwacji mało różni się od wartości średniej obserwowanej zmiennej
losowej.

Zatem w wypadku praw wielkich liczb rozkładem granicznym jest rozkład jednopunktowy (wartość średnia).

Centralne twierdzenia graniczne dotyczą zachowania się sum niezależnych zmiennych losowych, które przy dość ogólnych założeniach mają rozkład
asymptotycznie normalny. Twierdzenia te wyjaśniają zatem mechanizm częstego występowania w praktyce rozkładu normalnego (rozkładów zbliżonych do normalnego), 
gdyż na ogół obserwowana zmienna losowa jest sumą wielu niezależnych przyczyn (zmiennych) losowych.

Poniżej podamy przykłady twierdzeń z obu wymienionych grup.
\subsubsection{Prawa wielkich liczb}
Niech $X_1, X_2, \ldots$ będzie ciągiem niezależnych zmiennych losowych z wartością średnią $E(X_i)=m_i$ oraz wariancją $\mathscr{D}^2(X_i)=\sigma_i^2,
~i=1,2,\ldots$. Rozpatrzmy zmienną losową:
\[ M_n = \underbrace{\frac{\sum_{i=1}^nX_i}{n}}_\textrm{średnia arytmetyczna} \]
Oczywiście: 
\[ E(M_n) = \frac{1}{n} \sum_{i=1}^nm_i \]
($m_i$ -- wartość średnia kolejnych niezależnych zmiennych losowych)
oraz:
\[ \mathscr{D}^2(M_n) = \frac{1}{n^2} \sum_{i=1}^n\sigma_i^2 \]
\begin{theorem}[Słabe prawo wielkich liczb Markowa]
Jeśli przy powyższych założeniach i dodatkowo $\lim_{n\rightarrow\infty}\mathscr{D}^2(M_n)=0$, to:
\begin{equation}
\bigwedge_{\epsilon>0}\lim_{n\rightarrow\infty}P\left[|M_n-E(M_n)|\geq\epsilon\right]=0
\end{equation}
czyli ciąg zmiennych losowych $M_1,M_2,\ldots$ jest zbieżny według prawdopodobieństwa do zmiennej losowej $E(M_n)=\frac{1}{n}\sum_{i=1}^nm_i$.
\end{theorem}
Dowód (w domu): zastosuj nierówność Czebyszewa dla zmiennej losowej $M_n$.
\begin{theorem}[Słabe prawo wielkich liczb Chinczyna]
Niech $X_1,X_2,\ldots$ będzie ciągiem niezależnych zmiennych losowych o jednakowym rozkładzie prawdopodobieństwa z \\$E(X_i)=m,~i=1,2,\ldots$; 
wówczas:
\begin{equation}
\bigwedge_{\epsilon>0} \lim_{n\rightarrow\infty} P\left[|M_n-m|\geq\epsilon\right]=0
\end{equation}
czyli ciąg zmiennych losowych $M_1,M_2,\ldots$ jest zbieżny według prawdopodobieństwa do $m$.
\end{theorem}
\begin{theorem}[Mocne prawo wielkich liczb]
Niech $X_1,X_2,\ldots$ będzie ciągiem niezależnych zmiennych losowych o jednakowym rozkładzie prawdopodobieństwa z 
$E(X_i)=m, ~\mathscr{D}^2(X_i)=\sigma^2,~i=1,2,\ldots$; wówczas:
\begin{equation}
P(\lim_{n\rightarrow\infty}M_n=m)=1
\end{equation}
czyli ciąg zmiennych losowych $M_1,M_2,\ldots$ jest zbieżny z prawdopodobieństwem $1$ do wartości średniej $m$.
\end{theorem}
\subsubsection{Centralne twierdzenia graniczne}
Niech $X_1,X_2,\ldots$ będzie ciągiem niezależnych zmiennych losowych o jednakowym rozkładzie prawdopodobieństwa z $E(X_i)=m,
~\mathscr{D}^2(X_i)=\sigma^2\neq0,~i=1,2,\ldots$. Rozpatrzmy zmienną losową $Y_n=\sum_{i=1}^nX_i,~n=1,2,\ldots$. Unormujmy tę zmienną
losową wprowadzając zmienną losową $Z_n$ określoną wzorem:
\[ Z_n = \frac{Y_n-E(Y_n)}{\sigma_{Y_n}} \]
Oczywiście $E(Y_n)=n\cdot m$ oraz $\mathscr{D}^2(Y_n)=n\sigma^2$, a zatem:
\[ Z_n = \frac{Y_n-n\cdot m}{\sigma\sqrt{n}} \]
\begin{theorem}[Lindeberga-L\'evy'ego]
Przy powyższych założeniach ciąg zmiennych losowych $Z_1,Z_2,\ldots$ (tych unormowanych sum) jest zbieżny według dystrybuant do zmiennej
losowej $Z$ o rozkładzie $N(0,1)$ czyli:
\begin{equation}
\bigwedge_z \lim_{n\rightarrow\infty}F_n(z)=\frac{1}{\sqrt{2\pi}}\int_{-\infty}^ze^\frac{-z^2}{2}dz
\end{equation}
gdzie $F_n(Z)$ jest dystrybuantą zmiennej losowej $Z_n$.
\end{theorem}
Dowód otrzymujemy, obliczając funkcję charakterystyczną zmiennej losowej $Z_n$ i rozwijając ją w szereg Maclaurina.

\textbf{Uwagi}:
\begin{itemize}
 \item Szczególnym przypadkiem powyższego twierdzenia jest twierdzenie Moivre'a--Laplace'a, w którym zmienne losowe $X_i$ mają rozkład
zero-jedynkowy i mogą przyjmować dwie wartości: zero lub jeden. Wówczas $Y_n$ ma rozkład dwumianowy (na tej podstawie, dla dostatecznie 
dużych $n$ ($n>100$) rozkład dwumianowy można przybliżyć rozkładem normalnym).
\item W wypadku, gdy zmienne losowe $X_i$ nie mają identycznych rozkładów, warunek dostateczny prawdziwości tezy twierdzenia L-L podaje 
twierdzenie Lapunowa (o momentach), a warunek koniecznie-dostateczny twierdzenie Lindeberga-Fellera.
\end{itemize}

\section{Procesy losowe (stochastyczne)}
\subsection{Pojęcie procesu losowego i jego opis}
W rozdziale 2. rozpatrywaliśmy zmienne losowe, czyli wielkości, które zależały \textbf{tylko} od $\omega$, tzn \textbf{od przypadku}.
W praktyce spotykamy się na ogół z bardziej skomplikowanymi wielkościami losowymi, które zależą \textbf{zarówno od przypadku}, jak 
i~\textbf{od wartości pewnego parametru $t\in T$}. Oznacza to, że przypadkowy charakter tej wielkości zmienia się wraz ze zmianą parametru $t$.
Historycznie jednym z pierwszych przykładów takiej wielkości była każda współrzędna cząsteczki w tzw. ruchu Browna, która jest zmienną losową,
a ponadto zależy od czasu. Innym przykładem mogą być szumy zniekształcające sygnały radiowe, które są zmiennymi losowymi zależnymi także od
czasu. Również liczba pojazdów przejeżdżających przez dane skrzyżowanie, czy liczba procesów w danym systemie komputerowym są także zmiennymi
zależnymi od czasu. Podkreślmy, że parametrem, od którego zależą zmienne losowe nie musi być czas. Np. w ruchu turbulentnym prędkość cząsteczki 
cieczy jest zmienną losową (trójwymiarową) zależną także od punktu przestrzeni.

Rozszerzenie rachunku prawdopodobieństwa pozwalające badać takie wielkości losowe nazywa się \textbf{teorią procesów losowych} (przypadkowych,
stochastycznych). Zauważmy, że choć nie ma to znaczenia matematycznego, niektórzy autorzy rezerwują określenie ``proces'' dla sytuacji, gdy
wspomniany wyżej parametr jest czasem, a gdy nie, wówczas mówią o ``funkcjach losowych''. Nie będziemy czynili tego rozróżnienia, tym bardziej,
że w naszych zastosowaniach wzmiankowany parametr będzie zawsze czasem. 

Powyższe rozważania prowadzą do następujących definicji.
\begin{definition}
Procesem losowym nazywamy rodzinę zmiennych losowych $\left\{X_t(\omega),~t\in T\right\}$ określoną na przestrzeni probabilistycznej 
$\left\{ \Omega,~\mathscr{A},~P \right\}$.
\end{definition}
\lecture{09.11.2008}
Proces losowy jest zatem losową funkcją $t$, która dla każdego $t\in T$ jest zmienną losową. Tę zmienną losową $X_t(\omega)$ dla 
\textbf{ustalonego} $t$ nazywamy \textbf{wartością} procesu losowego w chwili $t$, a zbiór wszystkich wartości zmiennej losowej $X_t(\omega),~t\in T$ tworzy
\textbf{przestrzeń stanów procesu losowego}.

Jeśli zbiór $T$ jest skończony lub przeliczalny, to mówimy o \textbf{procesach losowych z czasem dyskretnym}, a jeśli $T$ jest zbiorem nieprzeliczalnym, to
o \textbf{procesach losowych z czasem ciągłym}. Zauważmy jeszcze, że w wypadku skończonego zbioru $T$ mamy w istocie do czynienia ze zmienną losową (wektorem
losowym), a w wypadku nieskończonego, przeliczalnego zbioru $T$ otrzymujemy ciąg zmiennych losowych.

W dalszym  ciągu będziemy zasadniczo rozpatrywali procesy z czasem ciągłym. 

W celu głębszego zrozumienia natury procesu losowego, spójrzmy nań jeszcze z drugiej strony. Jak pamiętamy z rozdziału 2. zmienna losowa przyporządkowywała
zdarzeniom elementarnym ogólnie punkty w przestrzeni $R^n$. W wypadku procesu losowego zdarzeniu elementarnemu jest przyporządkowywana skończona funkcja rzeczywista
$X(t)$ zwana \textbf{realizacją procesu losowego}. Zbiór realizacji procesu losowego odpowiadający wszystkim $\omega\in\Omega$ nazywać będziemy \textbf{przestrzenią realizacji}
procesu losowego. Prowadzi to do kolejnej definicji procesu losowego:
\begin{definition}
Procesem losowym nazywamy mierzalną względem $P$ transformacją przestrzeni zdarzeń elementarnych w przestrzeń realizacji.
\end{definition}
\textbf{Uwaga}:\\
Na przestrzeń realizacji narzucone są na ogół pewne dodatkowe wymagania np. gdyby była to przestrzeń Banacha (Stefan Banach, 1892-1945), czyli przestrzeń
unormowana (czyli też liniowa) i zupełna.

Przedstawiając obie definicje procesu losowego widzimy zatem, że jego pełne oznaczenie ma postać $\{X_t(\omega),~t\in T\}$ lub $X(\omega,t),~\omega\in\Omega,~t\in T$ (dla 
ustalonego $t$ - zmienna losowa, dla ustalonego $\omega$ - realizacja).
\begin{center}
\textbf{TODO}: wykres
\end{center}
Ponieważ jak pamiętamy z rozdziału 2., zależność od $\omega$ jako naturalną na ogół pomijamy, więc powyższe oznaczenia procesu losowego przyjmują postać
$\{X_t,~t\in T\},~X(t),~t\in T$. Dalej zbiór $T$ jest zwykle z góry określony i również możemy go pominąć, a ponadto zamiast $X_t$ wygodniej jest pisać $X(t)$.
Oznaczenie $X(t)$ może zatem dotyczyć całego procesu losowego, jego realizacji lub jego wartości w chwili $t$ (czyli zmiennej losowej). Z kontekstu musi jednoznacznie
wynikać, z którą sytuacją mamy do czynienia.

Przejdźmy obecnie do opisu procesu losowego.

Ponieważ dla ustalonego $t\in T$ proces losowy jest zmienną losową $X(t)$, więc jego pełny opis podaje dystrybuantę tej zmiennej losowej:
\[ F(x,t) = P[X(t)<x] \]
Rozkład ten nazywamy \textbf{jednowymiarowym rozkładem procesu losowego}. Oczywiście jednowymiarowy rozkład procesu losowego nie odzwierciedla zależności
między wartościami tego procesu dla różnych $t$, jest on zatem wystarczającym opisem procesu losowego tylko wtedy, gdy dla dowolnych układów chwil odpowiednie zmienne
losowe są niezależne, co w praktyce w zasadzie nie zachodzi.

Definiuje się zatem rozkład procesu losowego dla układu $n$ chwil.
\begin{definition}
$n$\dywiz wymiarowym rozkładem procesu losowego $X(t)$ nazywamy łączny rozkład wartości tego procesu dla układu chwil $t_1,t_2,...,t_n,~\bigwedge_it_i\in T$,
czyli łączny rozkład wektora losowego $[X(t_1), X(t_2),\ldots,X(t_n)]$ scharakteryzowany przez łączną dystrybuantę
\begin{equation}
F(x_1,t_1;~x_2,t_2;\ldots x_n,t_n) = P[X(t_1)<x_1,X(t_2)<x_2,\ldots,X(t_n)<x_n]
\end{equation}
\end{definition}
Z twierdzenia Kołmogorowa wynika, że znajomość $n$\dywiz wymiarowych rozkładów procesu losowego dla dowolnego, skończonego $n$ i dowolnego układu chwil
$t_1,t_2,\ldots,t_n,~\bigwedge_it_i\in T$ potrzeba i wystarcza do pełnego opisu procesu losowego.
\subsection{Momenty procesu losowego}
Podobnie jak w wypadku zmiennych losowych także dla procesów losowych definiuje się pewne ich charakterystyki prostsze od $n$\dywiz wymiarowych rozkładów, w szczególności
momenty.

Np. znając jednowymiarowy rozkład procesu losowego scharakteryzowany przez dystrybuantę $F(x,t)$, możemy zdefiniować wartość średnią i wariancję procesu losowego:
\begin{definition}
\textbf{Wartością średnią procesu losowego} $X(t)$ nazywamy funkcję $m(t)$ określającą dla każdego $t\in T$ wartość średnią zmiennej losowej $X(t)$, którą proces jest w chwili
$t$, czyli:
\begin{equation}
m(t)=E[X(t)]=\int_{-\infty}^\infty xdF(x,t)
\end{equation}
\end{definition}
\begin{definition}
\textbf{Wariancją procesu losowego} $X(t)$ nazywamy funkcję $\sigma^2(t)$ określającą dla każdego $t\in T$ wariancję zmiennej losowej $X(t)$, którą proces jest w chwili
$t$, czyli:
\begin{equation}
\sigma^2(t)=\mathscr D^2[X(t)]=E[X(t)-m(t)]^2
\end{equation}
\end{definition}
Oczywiście jednowymiarowe momenty procesu losowego nie charakteryzują zależności między jego wartościami w różnych czasowych, żeby je scharakteryzować weźmy najlepiej dwie
chwile $t_1,t_2,~t_i\in T,~i=1,2$.
\begin{definition}
 \textbf{Funkcję korelacyjną} procesu losowego $X(t)$ definiujemy wzorem:
\begin{equation}
R_X(t_1,t_2)=E\{[X(t_1)-m(t_1)][X(t_2)-m(t_2)]
\end{equation}
\end{definition}
\begin{definition}
\textbf{Unormowaną funkcję korelacyjną} lub \textbf{współczynnik korelacji procesu losowego} $X(t)$ definiujemy wzorem:
\begin{equation}
r_X(t_1,t_2)=\frac{R_X(t_1,t_2)}{\sigma(t_1)\sigma(t_2)}
\end{equation}
\end{definition}
Funkcję korelacyjną procesu losowego $X(t)$, podobnie jak jego inne wielowymiarowe momenty, możemy zdefiniować dla dowolnego $n$:
\[ R_X(t_1,t_2,\ldots,t_n) = E\{[X(t_1)-m(t_1)][X(t_2)-m(t_2)]\cdots[X(t_n)-m(t_n)] \]
Jest to \textbf{$n$\dywiz wymiarowa funkcja korelacyjna procesu losowego $X(t)$}. W praktyce okazuje się jednak, że znajomość dwóch pierwszych momentów procesu losowego
nam wystarcza. Ponieważ głównym z tych momentów jest funkcja korelacyjna, więc  teoria procesów losowych oparta na znajomości dwóch pierwszych momentów tych procesów nazywa 
się \textbf{teorią korelacyjną procesów} losowych. W niektórych wypadkach teoria ta jest ogólna np. dla tzw. procesów losowych gaussowskich, czyli takich, których wszystkie 
skończenie wymiarowe rozkłady są gaussowskie.

Często w praktyce zachodzi potrzeba łącznego badania kilku procesów losowych (np. w układach sterowania o wielu wejściach i wyjściach), prowadzi to do pojęcia 
\textbf{wektorowego procesu losowego}. Rozpatrzmy ten proces ograniczając się do $n=2[X(t),Y(t)]$. Zdefiniujmy moment charakteryzujący ten proces w dwóch różnych 
chwilach czasowych $t_1$ i $t_2$. 
\begin{definition}
\textbf{Funkcję korelacji wzajemnej} procesów losowych $X(t),Y(t)$ definiujemy wzorem:
\begin{equation}
R_{XY}(t_1,t_2)=E\left\{ [X(t_1)-m_X(t_1)][Y(t_2)-m_Y(t_2)] \right\}
\end{equation}
\end{definition}
W związku z tym, funkcję korelacyjną \textbf{pojedyńczego} procesu losowego nazywa się niekiedy \textbf{funkcją korelacji własnej} lub \textbf{funkcją autokorelacji}
i~oznacza np. $R_{XX}(t_1,t_2)$.

Macierz:
\begin{eqnarray}
\left[
\begin{array}{ll}
R_{XX}(t_1,t_2) & R_{XY}(t_1,t_2) \\\nonumber
R_{YX}(t_1,t_2) & R_{YY}(t_1,t_2)
\end{array}
\right]
\end{eqnarray}
nazywamy \textbf{macierzą korelacyjną} wektora losowego $[X(t),Y(t)]$.\\
Uwaga:

Mówiąc \textbf{proces losowy} mamy na uwadze rzeczywisty proces losowy. W ogólności rozpatruje się zespolone procesy losowe w postaci:
\[ X(t) = X_1(t) + iX_2(t) \]

\subsection{Procesy stacjonarne}
Ogólna teoria procesów losowych oparta na znajomości ich $n$\dywiz wymiarowych rozkładów dla dowolnego skończonego $n$ jest dla celów praktycznych zbyt skomplikowana.
Wyróżnia się zatem różne klasy procesów losowych, które tę teorię upraszczają. W dalszych punktach zdefiniujemy główne z tych klas, rozpoczynając od 
\textbf{procesów stacjonarnych}.

Wyróżnia się procesy stacjonarne w \textbf{sensie węższym} i w \textbf{sensie szerszym}.
\begin{definition}
 Proces losowy $X(t)$ nazywamy \textbf{stacjonarnym w węższym sensie}, jeśli dla dowolnego $n$, dowolnego układu chwil $t_1,t_2,\ldots,t_n$ oraz dowolnego
$\Delta$ takiego, że $(t_i+\Delta)\in T,~i=1,2,\ldots,n$ zachodzi:
\begin{equation}
F(x_1,t_1;~x_2,t_2;~\ldots;~x_n,t_n)=F(x_1,t_1+\Delta;~x_2,t_2+\Delta;~\ldots;~x_n,t_n+\Delta)
\end{equation}
($n$\dywiz wymiarowa dystrybuanta nie zmienia się w przesunięciu liniowym o $\Delta$; charakter losowy się nie zmienia!)
\end{definition}
\lecture{05.01.2009}
%----------------wklejone od Jarka ^-----------------------------------------------------------------------------------------------------------
W szczególności dla $n=1$:
\[ F(x,t) = F(x,t+\Delta) \]
A zatem:
\[ F(x,t) = F(x) \]
Dla $n=2$:
\[ F(x_1,t_1; x_2,t_2,) = F(x_1,t_1+\Delta;x_2,t_2+\Delta) \]
\begin{center}
(tu rysunek - oś)
\end{center}

A zatem:
\[ F(x_1,t_1; x_2,t_2) = F(x_1,x_2,\tau),~~\textrm{gdzie}~\tau=t_2-t_1. \] 
Widzimy zatem, że dla procesu stacjonarnego w węższym sensie, jego rozkład jednowymiarowy w ogóle nie zależy od $t$, a rozkład 
dwuwymiarowy zależy tylko od \textbf{różnicy} $\tau = t_2-t_1$. W konsekwencji dla procesu takiego, $m(t)$ nie zależy w ogóle od $t$ 
($m(t) = m$), oraz funkcja korelacyjna \[R_X(t_1,t_2) = R_X(\tau) = E\{[X(t)-m][X(t+\tau)-m]\}. \]
Oczywiście przy założeniu, że $R_X(\tau)$ istnieje.

Powyższa obserwacja prowadzi do definicji procesu stacjonarnego w szerszym sensie.
\begin{definition}
Proces losowy $X(t)$ nazywamy \textbf{stacjonarnym w szerszym sensie}, jeśli istnieją dla niego $m(t)$ i $R_X(t_1, t_2)$, oraz:
\[ m(t) = m = const \]
\[ R_X(t_1, t_2) = R_X(\tau),~~\tau = t_2-t_1\]
\end{definition}
Łatwo wykazać następujące twierdzenie:
\begin{theorem}
Proces stacjonarny w węższym sensie, dla którego wartość średnia z kwadratu jest skończona ($E[X^2(t)]<\infty$), jest stacjonarny w szerszym sensie.
Dla procesów normalnych (gaussowskich), słuszne jest również twierdzenie odwrotne.
\end{theorem}
\textbf{Uwaga:} niektórzy autorzy, mówiąc ``proces losowy stacjonarny'', mają na myśli proces stacjonarny w sensie węższym i szerszym.

Z pojęciem stacjonarności procesu losowego wiąże się pojęcie procesu o przyrostach stacjonarnych.
\begin{definition}
Proces losowy $X(t)$ nazywamy procesem o \textbf{przyrostach stacjonarnych} (\textbf{jednorodnych}), jeśli dla dowolnego $\Delta$ 
takiego, że $(t+\Delta)\in T$, proces losowy $Y(t) = X(t+\Delta) - X(t)$ jest stacjonarny w węższym sensie.
\end{definition}

\subsection{Procesy o przyrostach niezależnych i procesy Markowa}
\begin{definition}
Proces losowy $X(t)$ nazywamy procesem \textbf{o przyrostach niezależnych}, jeśli dla dowolnego $n$, dowolnego układu chwil $t_1<t_2<\ldots<t_n$, zmienne
losowe $X(t_1),~X(t_2)-X(t_1),~X(t_3)-X(t_2),~\ldots,~X(t_n)-X(t_{n-1}) $ są niezależne.
\end{definition}
Ważną klasą procesów o przyrostach niezależnych są procesy Poissona (będziemy je rozpatrywać na badaniach operacyjnych).
\begin{definition}
 Proces losowy $X(t)$ nazywamy \textbf{procesem Markowa}, jeśli dla dowolnego $n$, dowolnego układu chwil $t_1<t_2<\ldots<t_n$, oraz dla dowolnych liczb
rzeczywistych $x_1,~x_2,~\ldots,~x_n$ zachodzi:
\begin{eqnarray}
& P[X(t_n)<x_n|X(t_{n-1})=x_{n-1},~X(t_{n-2})=x_{n-2},~\ldots,~X(t_1)=x_1] \\\nonumber
& = P[X(t_n)<x_n|X(t_{n-1})=x_{n-1}]
\end{eqnarray}
\end{definition}
Widzimy zatem, że dla procesu Markowa rozkład warunkowy jego wartości w dowolnej chwili $t_n$, pod warunkiem, że jego wartości w chwilach $t_{n-1},~
t_{n-2},~\ldots$ są ustalone,
zależy tylko od wartości tego procesu w chwili $t_{n-1}$.
Innymi słowy, własności procesu Markowa w dowolnej chwili $t_n$ zależą tylko od jego wartości w chwili $t_{n-1}$, a nie zależą od jego wartości w chwilach
poprzedzających chwilę $t_{n-1}$.
Ta własność rozpatrywanego procesu nazywa się \textbf{własnością Markowa} (\textbf{własnością braku pamięci}).
Dla pełnego opisu procesu Markowa wystarczy zatem znajomość dystrybuanty warunkowej:
\[ F(x,y,s,t) = P[X(t)<x|X(s)=y],~s<t \]
Oczywiście dla pełnego opisu procesu Markowa, wystarcza także znajomość rozkładu wektora losowego $[X(s),~X(t)]$, wraz z tzw. 
\textbf{dystrybuantą początkową} $F(s,y)=P[X(s)<y]$.
A~zatem, dla pełnego opisu procesu Markowa, \textbf{wystarcza rozkład dwuwymiarowy}. 

Można wykazać następujące twierdzenie:
\begin{theorem}
Jeśli $P[X(t_1)=c]=1$, gdzie $c$ jest dowolną stałą, to proces o przyrostach niezależnych jest procesem Markowa. Twierdzenie odwrotne nie jest prawdziwe.
\end{theorem}
Zauważmy, że szczególna postać powyższego warunku to $P[X(0)=0]=1$.
\subsection{Ergotyczność procesów losowych}
\[ m(t) = E[X(t)] = \int_{-\infty}^{\infty}xdF(x,t) \]
Zauważmy, że w celu wyznaczenie momentów procesu losowego, wg definicji podanej w punkcie 2, czyli jako odpowiednik średnich po zbiorze, musielibyśmy
dysponować \textbf{jednocześnie} jego \textbf{wszystkimi} realizacjami, co w praktyce jest na ogół niemożliwe. Powstaje zatem naturalne pytanie:
jeśli znamy z eksperymentów \textbf{pojedynczą} realizację, to przy jakich założeniach można na tej podstawie wyznaczać momenty tego procesu, uśredniając
odpowiednio tę realizację \textbf{po czasie}? Opowiedź na to pytanie jest przedmiotem tzw. \textbf{twierdzeń ergodycznych}, a procesy losowe, 
dla których średnie po czasie z pojedynczej realizacji mogą być utożsamiane z odpowiednimi średnimi po zbiorze, nazywają się procesami \textbf{ergodycznymi
względem odpowiedniego momentu}, np. względem wartości średniej czy funkcji korelacyjej. Zauważmy, że twierdzenia ergodyczne w odniesieniu do wartości
 średniej są uogólnieniem praw wielkich liczb na nieprzeliczalną liczbę zmiennych losowych. 

Przykładowo, dla procesu stacjonarnego w szerszym sensie ergodyczność względem wartości średniej oznacza, że:
\[ m = E[X(t)] \approx X(t)= \frac{1}{T}\int_0^TX(t)dt \]
a względem funkcji korelacyjnej:
\[ R_X(\tau) = E\{[X(t)-m][X(t+\tau)-m]\} \approx \overline{[X(t)-m][X(t+\tau)-m]} \]
Można wykazać, że dla procesów takich z ciągłą funkcją korelacyjną, warunkiem wystarczającym ergodyczności względem powyższych elementów jest:
\[ \lim_{|\tau|\rightarrow\infty} R_X(\tau)=0 \]
co w praktyce jest zawsze spełnione.
\lecture{12.01.2009}
\section{Elementy statystyki matematycznej}
\subsection{Przedmiot statystyki matematycznej}
Głównym przedmiotem statystyki matematycznej jest tzw. wnioskowanie statystyczne, polegające na tym, żeby na podstawie przebadania odpowiednio wybranej
\textbf{części} elementów (tzw. próby) badanego zbioru (tzw. populacji generalnej) wyciągnąć matematycznie uzasadnione wnioski dotyczące \textbf{całości}
tego zbioru.

Zasadniczym aparatem formalnym, którym posługuje się statystyka matematyczna jest teoria prawdopodobieństwa. Można zatem powiedzieć, że statystyka 
matematyczna jest działem zastosowań teorii prawdopodobieństwa. Jest to jednak dział o tyle specyficzny, że jego obustronne związki z teorią prawdopodobieństwa są wyjątkowo ścisłe.

Historycznie, niektóre idee zastosowań rachunku prawdopodobieństwa wykorzystywane dziś w statystyce pochodzą od Bayesa, Laplace'a i Gaussa, jednak dopiero
 w XX wieku statystyka matematyczna przekształciła się w samodzielną dyscyplinę matematyki. Dokonało się to głównie dzięki Anglikom: Karlowi Pearsonowi,
Ronaldowi A. Fisherowi oraz Polakowi Jerzemu Spławie Neymanowi (1894-1981).
\subsection{Próba losowa}
Zacznijmy od pojęcia \textbf{próby} jako odpowiednio wybranej części elementów badanego zbioru (populacji generalnej). Załóżmy, że liczność wynosi $n$,
oraz że interesuje nas pojedyńcza cecha elementów populacji generalnej (np. wiek, wzrost, miesięczny zarobek mieszkańca Polski), czyli zmienna losowa
$X$. Ostatecznie chodzi nam zatem o ciąg liczb $(x_1,~x_2,~\ldots,~x_n)$ odpowiedających interesującej nas cesze $X$.

Próba winna być \textbf{reprezentatywna}, tzn. spełniać dwa postulaty:
\begin{enumerate}
\item[a)] jednakowej szansy trafienia do próby każdego elementu populacji generalnej;
\item[b)] dostatecznej liczności.
\end{enumerate}
W dalszym ciągu mówiąc ``próba'' będziemy mieli na myśli tzw. \textbf{próbę prostą}, która dodatkowo spełnia trzeci postulat:
\begin{enumerate}
\item[c)] niezależności  wyników badań elementów populacji generalnej trafiających do próby. 
\end{enumerate}
Realizacją postulatu b) zajmiemy się w p. 4.3.4. Realizacja postulatu a) i c) wymaga przestrzegania, by poszczególne badania elementów populacji generalnej trafiających do próby były wykonywane w tych samych warunkach niezależnie od siebie.

W wypadku, gdy populacja generalna jest \textbf{skończona}, to realizację postulatu c) zapewnia zastosowanie tzw. \textbf{schematu losowania ze
 zwracaniem}, w którym wylosowany element populacji generalnej jest do niej zwracany przed pobraniem kolejnego. Natomiast spełnienie postulatu a)
w tym wypadku można uzyskać jednym z poniższych sposobów:
\begin{itemize}
\item zapisanie numerów odpowiadających poszczególnym elementom populacji generalnej na kartkach i losowanie $n$ kartek;
\item zastosowanie tablic liczb losowych;
\item zastosowanie generatorów liczb pseudolosowych.
\end{itemize}
Ogólna teoria losowania jest oddzielnym działem statystyki matematycznej, omówiona w specyficznych monografiach.

W celu przeprowadzenia głębszych badań, musimy na próbę spojrzeć ogólniej, a mianowicie rozważyć \textbf{wszystkie możliwe próby o liczności n}, jakie
można uzyskać w danej populacji generalnej, stosując odpowiedni schemat losowania. Prowadzi to do pojęcia próby losowej.
\begin{definition}
\textbf{Próbą losową} nazywamy wektor losowy $(X_1,~X_2,~\ldots,~X_n)$, którego składowe są niezależnymi zmiennymi losowymi o tym samym rozkładzie 
prawdopodobieństwa, będącym rozkładem interesującej nas cechy $X$ (czyli zmiennej losowej) w populacji generalnej.
\end{definition}
\begin{center}
 (rysunek)
\end{center}
Rozpatrywane dotychczas próby $(x_1,~x_2,~\ldots,~x_n)$ są zatem realizacjami próby losowej $(X_1,~X_2,~\ldots,~X_n)$.

W dalszym ciągu będziemy stosowali terminu ``próba'' w obu powyższych znaczeniach, dbając o ty, by kontekst był jednoznaczny.
Ponieważ elementy próby losowej są niezależnymi zmiennymi losowymi, więc łączna dystrybuanta rozpatrywanego wektora losowego $(X_1,~X_2,~\ldots,~X_n)$
wyraża się wzorem:
\begin{eqnarray} 
F(x_1,x_2,\ldots,x_n) &=& P(X_1<x_1,X_2<x_2,\ldots,X_n<x_n)\\\nonumber
&=&\underbrace{P(X<x)\cdot P(X<x)\cdot \ldots\cdot P(X<x)}_{\textrm{n razy, z niezależności i tego samego rozkładu}}= F^n(x) 
\end{eqnarray}
gdzie $F(x)$ jest dystrybuantą interesującej nas cechy $X$. To samo dotyczy funkcji prawdopodobieństwa (czy funkcji gęstości prawdopodobieństwa).
Podamy teraz definicję statystyki.
\begin{definition}
\textbf{Statystyką} nazywamy dowolną funkcję próby losowej, czyli funkcję postaci $Y=f(X_1,X_2,\ldots,X_n),~f:R^n\longrightarrow R^1$.
\end{definition}
\subsection{Podstawy teorii estymacji}
\subsubsection{Przedmiot teorii estymacji}
Teoria estymacji jest działem statystyki matematycznej zajmującym się oceną parametrów rozkładu interesującej nas cechy  $X$ populacji generalnej, na 
podstawie wartości parametrów wyliczonych z prób. Wyróżnia się dwa główne działy teorii estymacji: estymację punktową i estymację przedziałową.
\begin{itemize}
 \item \textbf{estymacja punktowa} - wyznaczenie takiej liczby, która może być uznana za najlepszą ocenę wybranego parametru;
\item \textbf{estymacja przedziałowa} - wyznaczenie takiego losowego przedziału, do którego z zadanym prawdopodobieństwem należy nieznana
wartość parametru.
\end{itemize}
\subsubsection{Estymatory i ich klasyfikacja}
W dalszym ciągu mówiąc ``estymator'' bedziemy mieli na myśli estymator punktowy.

Załóżmy, że interesuje nas parametr $q$ rozkładu cechy $X$ w populacji generalnej, lub w maks. skrócie: parametr $q$ populacji generalnej. Oznaczmy przez
$F(x,q)$ dystrybuantę interesującej nas cechy $X$ populacji generalnej.
\begin{definition}
\textbf{Estymatorem} parametru $q$ cechy $X$ nazywamy dowolną statystykę $ Q_n = f(X_1,X_2,\ldots,X_n) $ taką, że:
\begin{equation}
\exists_{c>0}:\lim_{n\rightarrow\infty}P[|Q_n-q|<c]=1
\end{equation}
\end{definition}
Wartość estymatora $Q_n$ wyznaczoną z próby $(x_1,x_2,\dots,x_n)$, czyli $q_n=f(x_1,x_2,\dots,x_n)$ nazywamy \textbf{oceną} parametru $q$.

Z powyższej definicji widzimy, że może być wiele estymatorów danego parametru. Wymagana jest zatem ich klasyfikacja. W dalszym ciągu rozpatrzymy
główne klasy estymatorów, a mianowicie:
\begin{itemize}
 \item estymatory zgodne;
\item estymatory nieobciążone;
\item estymatory najefektywniejsze.
\end{itemize}
\begin{definition}
\textbf{Estymatorem zgodnym} parametru $q$ nazywamy estymator $Q_n$ tego parametru taki, że:
\begin{equation}
\forall_{\epsilon>0}:\lim_{n\rightarrow\infty}P[|Q_n-q|<\epsilon]=1.
\end{equation}
\end{definition}
Zatem jest to estymator stochastycznie zbieżny (wg prawdopodobieństwa) do parametru $q$.
\begin{definition}
\textbf{Estymatorem nieobciążonym} parametru $q$ nazywamy estymator $Q_n$ tego parametru taki, że:
\begin{equation}
E(Q_n) = q,~~n=1,2,\ldots.
\end{equation}
\end{definition}
Różnicę $E(Q_n)-q$ nazywamy obciążeniem estymatora $Q_n$.
\begin{definition}
\textbf{Estymatorem najefektywniejszym} parametru $q$ nazywamy estymator nieobciążony tego parametru $Q_n$, który ma minimalną wariancję
\end{definition}
Wyznaczenie estymatora najefektywniejszego, przy założeniu że on istnieje, umożliwia tzw. nierówność Rao\dywiz Cram\'{e}ra, zwana też nierównością informacyjną.

Można mianowicie wykazać, że jeśli $Q_n$ jest estymatorem nieobciążonym parametru $q$, to prawdziwa jest następująca \textbf{nierówność Rao-Cram\'{e}ra}:
\begin{eqnarray}
\mathscr{D}^2(Q_n)\geq \left\{ \begin{array}{ll}
\frac{1}{nE\left[ \frac{\partial \ln F(x,q)}{\partial q} \right]^2}, & \textrm{gdy cecha $X$ jest zmienną losową typu ciągłego}\\ \\
\frac{1}{nE\left[ \frac{d\ln p_i(q)}{dq} \right]^2}, & \textrm{gdy cecha $X$ jest zmienną losową typu skokowego}
\end{array} \right.
\end{eqnarray}
Prawa strona nierówności Rao\dywiz Cram\'{e}ra nazywa się \textbf{informacją Fishera} zawartą w próbie. Wyzej podany fakt można zatem wysłowić w ten
 sposób, że wariancja nieobciążonego estymatora parametru $q$ nie może być mniejsza od informacji Fishera.

\lecture{19.01.2009}
\subsubsection{Ogólne metody wyznaczania estymatorów}
Rozważmy dwa bardzo ważne przykłady wyznaczania estymatorów.

\textbf{Przykład 1} - estymacja wartości średnich\\
Chcemy estymować wartość średnią $E(X)=m$ cechy $X$ populacji generalnej. W tym celu pobieramy próbę $(x_1,~x_2,~\ldots,~x_n)$ i wyznaczamy z niej
średnią arytmetyczną $\overline{x}=\frac{1}{n}\sum_{i=n}^nx_i$, którą traktujemy jako ocenę $m_n$ wartości średniej $m$, czyli jako realizację estymatora
wartości średniej, danego wzorem:
\[ M_n = \frac{1}{n}\sum_{i=1}^nX_i \]
gdzie $(X_1,~X_2,~\ldots,~X_n)$ jest próbą losową. Zbadajmy własności tego estymatora. 

Po pierwsze łatwo zauważyć, że spełnione są w odniesieniu do niego
wszystkie założenia wielkich liczb Chinczyna, a zatem estymator $M_n$, jako stochastycznie zbieżny do $E(X)=m$, jest estymatorem zgodnym parametru
$m$. 

Po drugie, ponieważ:
\[ E(M_n) = E\left(\frac{1}{n}\sum_{i=1}^nX_i\right) = \frac{1}{n}\sum_{i=1}^nE(X_i) = \frac{n\cdot m}{n} = m,~~n=1,2,\ldots \]
jest to również estymator nieobciążony.

Zbadajmy wreszcie, czy jest to estymator najefektywniejszy. W tym celu musimy przyjąć założenie o rozkładzie cechy $X$ populacji generalnej, gdyż chcemy
skorzystać z nierówności Rao\dywiz Cram\'{e}ra. Załóżmy zatem, że jest to rozkład normalny $N(m,\sigma)$, czyli:
\[ f(x,m) = \frac{1}{\sigma\sqrt{2\pi}}\exp\left(-\frac{(x-m)^2}{2\sigma^2}\right). \]
Mamy zatem:
\[ \ln f(x,m) = -\ln\sigma\sqrt{2\pi}-\frac{(x-m)^2}{2\sigma^2},\]
stąd:
\[ \frac{\partial\ln f(x,m)}{\partial m} = \frac{x-m}{\sigma^2}. \]
Podstawiając to do prawej strony nierówności Rao\dywiz Cram\'{e}ra otrzymujemy:
\[ \frac{1}{nE\left( \frac{(X-m)^2}{\sigma^4}\right)} = \frac{1}{n\frac{\sigma^2}{\sigma^4}} = \frac{\sigma^2}{n}. \]
Obliczmy teraz lewą stronę nierówności Rao\dywiz Cram\'{e}ra, czyli $\mathscr{D}^2(M_n)$:
\[ \mathscr{D}^2(M_n) = \mathscr{D}^2\left(\frac{1}{n}\sum_{i=1}^nX_i\right) = \frac{1}{n^2}\sum_{i=1}^n\mathscr{D}^2(X_i) = 
\frac{n\sigma^2}{n^2} = \frac{\sigma^2}{n}. \]
A zatem nierówność Rao\dywiz Cram\'{e}ra jest spełniona, czyli (przy założeniu, że cecha $X$ ma rozkład normalny) estymator $M_n$ jest estymatorem
najefektywnieszym parametru $m$. 

\textbf{W domu}: pokazać to samo przy założeniu, że $X$ ma:
\begin{enumerate}
 \item rozkład Poissona
\item rozkład dwumianowy
\end{enumerate}

\textbf{Przykład 2} - estymacja wariancji\\
Załóżmy, że chcemy estymować wariancję $\mathscr{D}^2(X)=\sigma^2$ cechy $X$ populacji generalnej, której \textbf{wartość średnia nie jest znana}.
W tym celu pobieramy próbę $(x_1,~x_2,~\ldots,~x_n)$ i wyliczamy z niej średnią arytmetyczną:
\[ \overline{(x_i-\overline{x})^2} = \frac{1}{n}\sum_{i=1}^n (x_i-\overline{x})^2, ~~\textrm{gdzie}~\overline{x}=\frac{1}{n}\sum_{i=1}^nx_i, \]
którą traktujemy jako ocenę $\sigma_n^2$ wariancji $\sigma^2$, czyli jako realizację estymatora wariancji danego wzorem:
\[ S_n^2 = \frac{1}{n}\sum_{i=1}^n(X_i-M_n)^2, \]
gdzie $M_n$ jest estymatorem wartości średniej, omówionej w przykładzie 1; $(X_1,~X_2,~\ldots,~X_n)$ oznacza próbę losową. 

Zbadajmy własności tego estymatora. Zacznijmy od zgodności i rozpatrzmy zmienne losowe $Z_i = (X_i-M_n)^2,~i=1,2,\ldots,n$. Widzimy, że
estymator $S_n^2=\frac{1}{n}\sum_{i=1}^nZ_i$ jest średnią arytmetyczną niezależnych zmiennych losowych o tym samym rozkładzie prawdopodobieństwa:
\[ E(Z_i) = \sigma^2,~~i=1,2,\ldots,n. \]
Spełnione są zatem wszystkie założenia słabego prawa wielkich liczb Chinczyna, czyli nasz estymator, jako stochastycznie zbieżny do $E(Z)=\sigma^2$,
jest estymatorem zgodnym wariancji $\sigma^2$.

Zbadajmy, czy jest to również estymator nieobciążony tego parametru:
\begin{eqnarray}
\nonumber E(S_n^2) &=& E\left(\frac{1}{n}\sum_{i=1}^n(X_i-M_n)^2\right) = E\left(\frac{1}{n}\sum_{i=1}^n(X_i-m+m-M_n)^2\right) =\\\nonumber
&=& E\left(\frac{1}{n}\sum_{i=1}^n((X_i-m)-(M_n-m))^2\right) = \textrm{\textbf{dokończyć}}
\end{eqnarray}
Korzystając z faktu, iż $\mathscr{D}^2(M_n)=\frac{\sigma^2}{n}$, otrzymujemy:
\[ E(S_n^2) = \frac{n-1}{n}\sigma^2. \]
A zatem nasz estymator jest estymatorem obciążonym. Widzimy jednak, że ${S_n^2}'$ dany jest wzorem:
\[ {S_n^2}' = \frac{n}{n-1}S_n^2=\frac{1}{n-1}\sum_{i=1}^n(X_i-M_n)^2 \]
o wartościach ${\sigma_n^2}'=\frac{1}{n-1}\sum_{i=1}^n(x_i-\overline{x})^2$. 
Niestety, estymator ${S_n^2}'$ nie jest estymatorem najefektywniejszym wariancji dla nietrywialnych rozkładów cechy $X$.
\subsubsection{Estymacja przedziałowa}
Do tej pory zajmowaliśmy się estymatorami, których  realizację wyznaczamy z prób $(x_1,~x_2,~\ldots,~x_n)$; stanowiły ocenę nieznanej wartości parametru
$q$ populacji generalnej. Były to jak pamiętamy estymatory punktowe.

Teraz postawimy problem dokładności tej estymacji, czyli problem wyznaczania takiego losowego przedziału, którego realizacje pokrywają z zadanym
prawdopodobieństwem $p_u$ nieznaną wartość parametru $q$ populacji generalnej. Przedział ten nazywa się \textbf{przedziałem ufności} (lub 
\textbf{estymatorem przedziałowym}), a prawdopodobieństwo $p_u$ nazywa się \textbf{poziomem ufności}. Koncepcję estymacji przedziałowej zawdzięczamy
Jerzemu Spławie Neymanowi.

Przeanalizujmy ten problem na poziomie estymowanej wartości średniej $m$. Zacznijmy od wyznaczenia prawdopodobieństwa tego, że estymator $M_n$ przyjmie
wartość z zadanego przedziału $[a,b],~a<b,$ \textbf{przy założeniach}, że:
\begin{enumerate}
 \item[(1)] cecha $X$ ma rozkład $N(m,\sigma)$;
\item[(2)] $\sigma$ jest znana.
\end{enumerate}
Zauważmy najpierw, że ponieważ:
\begin{itemize}
 \item estymator $M_n$ jest nieobciążony,
\item rozkład normalny jest nieskończenie podzielny,
\item $\sigma_{M_n} = \frac{\sigma}{\sqrt{n}}$
\end{itemize}
przy założeniu (1) $M_n$ ma rozkład $N(m,\sigma_{M_n})$, gdzie $\sigma_{M_n}=\frac{\sigma}{\sqrt{n}}$,
a zatem:
\[ P(M_n\in [a,b]) = F(b)-F(a) = \frac{1}{\sigma_{M_n}\sqrt{2\pi}}\int_a^b\exp\left(-\frac{(m_n-m)^2}{2\sigma_{M_n}^2} \right)dm_n. \]
W celu wyliczenia tej całki, unormujmy zmiennej losowe, wprowadzając zmienną losową $Y$:
\[ Y = \frac{M_n-m}{\sigma_{M_n}} \]
Mamy 
\[ P(M_n\in [a,b])=P(Y\in[y_1,y_2])=F(y_2)-F(y_1) = \frac{1}{2\sqrt{\pi}}\int_{y_1}^{y_2}e^{-\frac{y^2}{2}}dy = p_u >0 \]
gdzie: 
\[y_1=\frac{a-m}{\sigma_{M_n}},~~~y_2=\frac{b-m}{\sigma_{M_n}}.\]
\lecture{26.01.2009}
Postawmy obecnie problem wyznaczania przedziału, który z zadanym prawdopodobieństwem $p_u$ pokryje nieznaną wartość $m$. Łatwo zauważyć, że w ogólności
problem ten nie ma jednoznacznego rozwiązania, gdyż otrzymujemy jedno równanie z dwiema niewiadomymi:
\[ F(y_2) - F(y_1) = p_u. \]
Przyjmując jednak naturalne w tym wypadku założenie, że $y_2=-y_1=y$
\begin{center}
 \textbf{*Tu wykres z tablicy*}
\end{center}
otrzymujemy:
\[ p_u = 1-2\underbrace{P(Y\geq y)}_{1-F(y)} = 1-2[1-F(y)], \]
stąd:
\[ F(y) = \frac{p_u+1}{2}. \]
Przyjmijmy, że $\frac{p_u+1}{2}=1-\frac{\alpha}{2}$, czyli $p_u = 1-\alpha$.
\begin{center}
 \textbf{*Tu wykres z tablicy*}
\end{center}
Oznaczmy przez $y_{\frac{\alpha}{2}}$ argument dystrybuanty $F(y)$ spełniający powyższy warunek, tzn. $F(y)=1-\frac{\alpha}{2}$. Wartość tę znajdujemy,
korzystając ``w tył'' z tablic rozkładu normalnego $N(0,1)$. Otrzymujemy zatem:
\begin{eqnarray}
\nonumber p_u & = & 1-\alpha = P\{Y\in [-y_{\frac{\alpha}{2}},y_{\frac{\alpha}{2}}]\} = P\left( -y_{\frac{\alpha}{2}} \leq \frac{M_n-m}{\sigma_{M_n}}
\leq y_{\frac{\alpha}{2}}\right) = \\\nonumber
& = & P\left(M_n - y_{\frac{\alpha}{2}}\sigma_{M_n} \leq m \leq M_n + y_{\frac{\alpha}{2}}\sigma_{M_n}\right) = P(|M_n-m|\leq d),
\end{eqnarray}
gdzie:
\[ d = y_{\frac{\alpha}{2}}\sigma_{M_n} = y_{\frac{\alpha}{2}}\frac{\sigma}{\sqrt{n}}. \]
Otrzymaliśmy zatem przedział o losowych końcach i zdeterminowanej długości $d$, którego realizacje z zadanym prawdopodobieństwem $p_u$ pokrywają
nieznaną wartość parametru $m$ populacji generalnej. Jest to zatem przedział ufności (estymator przedziałowy) dla estymatora $M_n$, przy rozpatrywanych
założeniach.

Przejdźmy obecnie do założeń, któreśmy poczynili, usuwając najpierw drugie z nich. W praktyce bowiem $\sigma^2$ również nie jest znana i musi być
estymowana. Jeśli w tym celu zastosujemy estymator ${S_n^2}'$, to można wykazać (zrobił to anglik W. S. Gosset ps. ``Student'', ur. 1908), że 
zmienna losowa $T = \frac{M_n-M}{\frac{{S_n}'}{\sqrt{n}}} $ ma rozkład, który nie zależy od $\sigma$. Jest to tzw. \textbf{rozkład t-Studenta}
o $(n-1)$ stopniach swobody, który jest ztablicowany. Ma zbliżoną postać do rozkładu normalnego, nie ma punktów przegięcia. Ponieważ jest to rozkład
symetryczny, więc wzór na przedział ufności (estymator przedziałowy) ma w tym wypadku postać:
\begin{eqnarray}
\nonumber p_u &=& 1-\alpha = P\left\{ T\in[-t_{\frac{\alpha}{2}}, t_{\frac{\alpha}{2}}]\right\} = P\left( -t_{\frac{\alpha}{2}} \leq \frac{M_n-m}{\frac{{S_n}'}{\sqrt{n}}} \leq t_{\frac{\alpha}{2}} \right) = \\\nonumber
&=& P\left(M_n - t_{\frac{\alpha}{2}}\frac{{S_n}'}{\sqrt{n}} \leq m \leq M_n + t_{\frac{\alpha}{2}}\frac{{S_n}'}{\sqrt{n}} \right) = P(|M_n-m|\leq D),
\end{eqnarray}
gdzie $t_{\frac{\alpha}{2}}$ znajdujemy ``w tył'' z tablic t-Studenta o $(n-1)$ stopniach swobody, jako argument dystrybuanty zmiennej losowej $T$,
spełniającej warunek $F(t) = 1-\frac{\alpha}{2}$, natomiast $D=t_{\frac{\alpha}{2}}\frac{{S_n}'}{\sqrt{n}}$.
Widzimy zatem, że w tym wypadku przedział ufności ma zarówno losowe końce jak i długość $D$.
\textbf{Uwaga}: jeśli $n\geq30$, to możemy w przybliżeniu skorzystać z tablic rozkładu $N(0,1)$.

Przejdźmy obecnie do pierwszego z naszych założeń i przyjmijmy, że nie jest ono prawdziwe. Możemy wówczas postawić hipotezę nieparametryczną o kształcie
rozkładu cechy $X$ populacji generalnej i zweryfikować ją na podstawie niezgodności.

Jeśli okaże się, że rozkład ten można uznać za normalny na tym poziomie istotności, to korzystamy z rozkładu t-Studenta. Jeśli jednak tak nie jest, a~mamy
 możliwość pobrania tzw. ``dużej próby'', to na podstawie twierdzenia Lindeberga-L\'{e}vy'ego możemy w przybliżeniu korzystać z tablic rozkładu $N(0,1)$.
Określenie ``duża'' w odniesieniu do estymacji wartości średniej i wariancji oznacza $n\geq30$.

\subsubsection{Minimalna liczność próby}
W~poprzednim punkcie pokazaliśmy, że mając dane $n$ oraz $p_u$ możemy (przy określonych założeniach) znaleźć przedział ufności, czyli dokładność estymacji.
Postawmy teraz problem odwrotny: chcemy wyznaczyć $n_{MIN}$, które dla zadanego $p_u$ zapewni zadaną dokładność estymacji (w praktyce, dokładność
nie gorszą od zadanej). Rozpatrzmy ten problem jak poprzednio, dla estymacji wartości średnich.

Jak pamiętamy, jeśli cecha $X$ ma rozkład $N(m,\sigma)$ i $\sigma$ jest znane, to dokładność estymacji wyraża się wzorem 
$d=y_{\frac{\alpha}{2}}\frac{\sigma}{\sqrt{n}}$. A zatem (z dokładnością do zaokrągleń):
\[ n_{MIN} = \left[ \frac{y_{\frac{\alpha}{2}}\cdot \sigma}{d} \right]^2, \]
gdzie $y_{\frac{\alpha}{2}}$ jest argumentem dystrybuanty $N(0,1)$, spełniającym warunek: 
\[F(y)=1-\frac{\alpha}{2},~~1-\alpha=p_u~~(\alpha=1-p_u).\]
Jeśli cecha $X$ ma rozkład $N(m,\sigma)$, ale $\sigma$ nie jest znane, to jeśli zastosujemy estymator ${S_n^2}'$, otrzymujemy:
\[ n_{MIN} = \left( \frac{t_{\frac{\alpha}{2}}\cdot {\sigma_n}'}{d} \right)^2,~~(d\textrm{~-- realizacja $D$})\]
gdzie $t_{\frac{\alpha}{2}}$ jest argumentem dystrybuanty rozkładu t-Studenta o $(n-1)$ stopniach swobody, spełniającym warunek: 
\[F(t)=1-\frac{\alpha}{2},~~\alpha=1-p_u.\]
Jeśli $n\geq30$, to możemy korzystać w przybliżeniu z tablic rozkładu $N(0,1)$ \textbf{niezależnie od rozkładu cechy $X$ populacji generalnej}.

\textbf{Uwaga}: jeśli w celu estymacji wariancji zastosujemy estymator obciążony $S_n^2$, to rozkład t-Studenta o $(n-1)$ stopniach swobody ma zmienna
losowa $T = \frac{M_n-m}{\frac{S_n}{\sqrt{n-1}}}$. Zauważmy jednak, że wartość $\sigma'$ nie jest znana przed pobraniem próby. Problem ten możemy 
rozwiązać na dwa sposoby.

Pierwszy sposób polega na pobraniu wstępnej próby dla wyznaczenia tej wartości, a dopiero potem ostatecznej próby o wyliczonej minimalnej liczności.

Drugi sposób na przyjęciu założenia, że estymujemy wariancję z dokładnością $d=\frac{{\sigma_n}'}{4}$. Wówczas:
\[ n_{MIN} = \left( \frac{t_{\frac{\alpha}{2}}\cdot {\sigma_n}'}{\frac{{\sigma_n}'}{4}} \right)^2 = 16t_{\frac{\alpha}{2}}^2. \]
Ponieważ już dla $p_u=0.9$ otrzymujemy  tu $n_{MIN}>30$, więc wyznaczając $t_{\frac{\alpha}{2}}$ (a raczej $y_{\frac{\alpha}{2}}$), możemy skorzystać
z tablic rozkładu normalnego $N(0,1)$.
\subsection{Podstawy teorii weryfikacji hipotez statystycznych}
\subsubsection{Podstawowe pojęcia}
Hipotezą statystyczną nazywamy każdy sąd o populacji generalnej, wydany bez przebadania całej tej populacji. Sądy dotyczące parametrów cech populacji
generalnej nazywamy \textbf{hipotezami parametrycznymi}, natomiast sądy dotyczące postaci rozkładu (kształtu) cech nazywamy 
\textbf{hipotezami nieparametrycznymi}. Sprawdzenie hipotez statystycznych nazywa się ich \textbf{weryfikacją}, a sposoby tej weryfikacji nazywają się
\textbf{testami statystycznymi}. Testy służące do weryfikacji hipotez parametrycznych nazywają się \textbf{testami parametrycznymi}, a testy służące
do weryfikacji hipotez nieparametrycznych \textbf{testami nieparametrycznymi} (w szczególności \textbf{testy zgodności}). Jeśli stawiamy tylko jedną 
hipotezę i celem testu jest stwierdzenie, czy należy ją odrzucić, to mówimy o tzw. \textbf{teście istotności}. Oczywiście testy nie dają pewności, czy
hipoteza jest prawdziwa, ale pozwalają sprawdzić prawdziwość hipotezy z prawdopodobieństwem dowolnie bliskim 1.
\subsubsection{Weryfikacja hipotez parametrycznych}
Pobierzmy $\alpha$ w ten sposób, że zdarzenia o prawdopodobieństwie zajścia $\leq\alpha$ uznajemy za mało prawdopodobne. Liczbę $\alpha$ nazywamy
\textbf{poziomem istotności hipotezy} i zwykle przyjmujemy z przedziału $[0.001, 0.005]$.

Jako przykład rozpatrzmy hipotezę parametryczną $H_0(m=m_0)$ przy założeniu, że cecha $X$ ma rozkład $N(m,\sigma)$, przy czym $\sigma$ nie jest znane
(jest estymowane za pomocą estymatora ${S_n}'$). \textbf{Uwaga}: oznaczenie $H_0$ wynika stąd, ze oryginalna hipoteza, którą stawiamy, nazywa się
\textbf{hipotezą zerową}.

Oznaczmy przez $t$ taką liczbę zależną od $\alpha$, która przy założeniu, że hipoteza $H_0(m=m_0)$ jest prawdziwa, spełnia warunek:
\[  P\left[ \left|\frac{M_n-m_0}{\underbrace{{S_n}'/\sqrt{n}}_{T}}\right|\geq t \right] = \alpha. \]
Jak wiemy, $t$ wyznaczamy z tablic t-Studenta o $(n-1)$ stopniach swobody (jeśli $n\geq30$ to możemy korzystać z tablic $N(0,1)$). Pobierzmy
próbę $n$\dywiz elementową i wyznaczmy z niej $m_n$ oraz ${\sigma_n}'$. Jeśli zatem
\[ |m_n-m_0|\geq t\cdot\frac{{\sigma_n}'}{\sqrt{n}}, \]
to hipotezę $H_0$ odrzucamy na poziomie istotności $\alpha$, gdyż zaszło zdarzenie mało prawdopodobne.

\textbf{Przykład}. \textit{Partia pudełek zapałek liczy $100000$ sztuk. Dostawca twierdzi, że w pudełku znajduje się średnio $54$ zapałki. Przeprowadzić
 test istotności dla hipotezy $H_0(m=54)$. Poziom istotności $\alpha=0.002$.} Ponieważ nie wiemy, jaki jest rozkład liczby zapałek w~pudełkach
populacji generalnej, a łatwo pobrać dużą próbę, to przyjmijmy $n=100$, co pozwoli nam z dużą dokładnością skorzystać z rozkładu $N(0,1)$. Pobierzmy zatem
 próbę $100$\dywiz elementową i~załóżmy, że wyznaczyliśmy z niej $m_n=51.4$, a ${\sigma_n}'=2.45$.

Musimy zatem wyznaczyć takie $t$ (a raczej $y$), że:
\[ P\left[ \left|\frac{M_n-54}{\underbrace{{\sigma_n}'/\sqrt{n}}_{0.245}}\right|\geq y \right] = 0.002 \]
\[ P(|Y|\geq y) = 0.002 \Longleftrightarrow P(Y<y)=0.001 \]
\begin{center}
 \textbf{*Dokończyć*}
\end{center}
Jeśli $|m_n-m_0|<t\cdot{\sigma_n}'/\sqrt{n}$, to nie oznacza to, że możemy hipotezę $H_0$ przyjąć. Oznacza to tylko tyle, że nie ma podstaw do jej
odrzucenia na poziomie istotności $\alpha$. Żeby przyjąć hipotezę, musimy postawić tzw. \textbf{hipotezę alternatywną} $H_1$, czyli taką, którą
skłonni jesteśmy przyjąć w wypadku odrzucenia hipotezy $H_0$, i zweryfikować ją na poziomie istotności $\beta$; np. w wypadku naszej hipotezy $H_0$
hipoteza $H_1$ miałaby postać $H_1(m=m_1),~m_1 \neq m_o$.
% \subsubsection{Weryfikacja hipotez nieparametrycznych}
\listoflectures
\end{document}
